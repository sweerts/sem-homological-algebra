\documentclass[a4paper,11pt]{article}
\usepackage[utf8]{inputenc}


%%%%%%%%%%%%%%%%%%%%%%%%%%%%%%%%%%%%%%%%%%%%%%%%%%%%%%%%%%%%%%%%%%%%%%%%%%
% Hier kommen Einstellungen

% Seitenränder
\usepackage{geometry}
\geometry{
	a4paper,
	left=30mm,
	right=30mm,
	top=35mm,
	bottom=35mm,
}

\usepackage{amssymb, amsfonts, amsbsy, amsmath, amsthm}
\usepackage[english]{babel}
\usepackage{enumitem}

% Bibliographie
\usepackage[german=quotes]{csquotes}
\usepackage[backend=biber, style=alphabetic, url=false, isbn=false, maxbibnames=99, giveninits=true]{biblatex}
\renewbibmacro{in:}{}
\addbibresource{literatur.bib}

% Graphik und kommutative Diagramme
\usepackage{graphicx}
\usepackage{tikz}
\usepackage[all,cmtip]{xy}

% Etwas Farbe, man kann sogar per rgb Wert definieren
\usepackage{color}
\newcommand{\black}{\color{black}}
\newcommand{\red}{\color{red}}
\newcommand{\green}{\color{dgreen}}
\newcommand{\blue}{\color{blue}}
\newcommand{\magenta}{\color{magenta}}
\definecolor{darkblue}{rgb}{0,0,.5}

% Das macht, dass man im pdf auf die Verweise klicken kann
\usepackage[colorlinks,linkcolor=darkblue,citecolor=darkblue,urlcolor=darkblue]{hyperref}

% Fuer Randbemerkungen
\usepackage{marginnote}
\usepackage[color]{changebar}
\changebarsep0.2cm
\changebarwidth0.03cm
\cbcolor{red}


% Ringe und Körper
\newcommand{\N}{{\mathbb N}}
\newcommand{\Z}{{\mathbb Z}}
\newcommand{\Q}{{\mathbb Q}}
\newcommand{\R}{{\mathbb R}}
\newcommand{\C}{{\mathbb C}}

% Potenzmenge
\renewcommand{\P}{{\mathfrak P}}

% Abbildungen
\newcommand{\id}{\operatorname{id}}
\newcommand{\Abb}{\operatorname{Abb}}

% Kategorien
\newcommand{\cat}[2]{\operatorname{#1}(#2)}

% Theorem Definitionen mit durchlaufender Numerierung
%\swapnumbers
\theoremstyle{definition}
\newtheorem{definition}{Definition}[subsection]
%\theoremstyle{plain}
\newtheorem{remark}{Remark}[subsection]
\newtheorem{example}{Example}[subsection]
\newtheorem{proposition}{Proposition}[subsection]
\newtheorem{lemma}{Lemma}[subsection]
\newtheorem{theorem}{Theorem}[subsection]
\newtheorem{algorithm}{Algorithm}[subsection]
\newtheorem{corollary}{Corollary}[subsection]
\newtheorem{assumption}{Assumption}[subsection]
\newtheorem{axiom}{Axiom}[subsection]
\newtheorem{exercise}{Exercise}[subsection]

% Beweise
%\renewcommand{\proofname}{Proof}

\title{Derived Functors}
\author{Steffen Weerts}
\date{\today}

%%%%%%%%%%%%%%%%%%%%%%%%%%%%%%%%%%%%%%%%%%%%%%%%%%%%%%%%%%%%%%%%%%%%%%%%%%
% Jetzt kommt der Text

\begin{document}
	\maketitle
	
	\section{Introduction}
This talk is based on \cite{Wei94}.

\begin{note}[Snake Lemma]\label{snake_lemma}
	Let
	\begin{align*}
		\begin{CD}
			0 @>>> A @>f>> B @>g>> C @>>> 0 \\
			@.   @VaVV   @VbVV    @VcVV    @. \\
			0 @>>> A' @>f'>> B' @>g'>> C' @>>> 0
		\end{CD}
	\end{align*}
	be a commutative diagram with exact rows. \\
	Then there is an exact sequence
	$$0 \rightarrow \ker(a) \rightarrow \ker(b) \rightarrow \ker(c) \rightarrow \coker(a) \rightarrow \coker(b) \rightarrow \coker(c) \rightarrow 0.$$
\end{note}

\begin{definition}[Mapping Cone]
	Let $f:B \rightarrow C$ be a map of chain complexes.
	The \textit{mapping cone} of $f$ is the chain complex whose degree $n$ part is $B_{n-1} \oplus C_n$, i. e.
	\[\dots \rightarrow B_{n} \oplus C_{n+1} \rightarrow B_{n-1} \oplus C_n \rightarrow B_{n-2} \oplus C_{n-1} \rightarrow \dots\]
	The differential is given by
	\[d(b,c) = (-d(b), d(c) - f(c)), \quad (b\in B_{n-1}, c\in C_n).\]
	There is a short exact sequence $0 \rightarrow C \rightarrow \cone(f) \rightarrow B[-1] \rightarrow 0.$ \\
	The mapping cone of $id_C$ (denoted by $\cone(C)$) is \textit{split exact}, i. e.
	\begin{enumerate}[label=(\roman*)]
		\item there are maps $s_n:C_n\rightarrow C_{n+1}$ such that $d=dsd$ ($C$ is \textit{split}),
		\item $C$ is exact as a sequence (\textit{acyclic}).
	\end{enumerate} 
\end{definition}

\begin{note}[Fundamental Theorem on Homomorphisms] \label{FTH}
	Let $\cat{A}$ be an abelian category. Let $f:A \rightarrow B$ be a morphism in $\cat{A}$.
	Then there is an isomorphism $A / \ker(f) \cong \im(f)$.
	Furthermore, if $f$ is an epimorphism, then $A / \ker(f) \cong B$.
\end{note}

\begin{note}[Zorn's Lemma] \label{zorn}
	 Suppose a partially ordered set $P$ has the property that every chain in $P$ has an upper bound in $P$. Then the set $P$ contains at least one maximal element. 
\end{note}
	\section{Derived Functors}

%\subsection{$\delta$-Functors}

\begin{notation}
	In this talk, we will use $\mathcal{A}$ and $\mathcal{B}$ as names for two arbitrary abelian categories.
\end{notation}

\begin{definition}
	A (covariant) homological $\delta$-functor between $\mathcal{A}$ and $\mathcal{B}$ is a collection of additive functors $T_n:\mathcal{A}\rightarrow\mathcal{B}$ for $n\geq0$, together with morphisms
	$$\delta_n:T_n(C)\rightarrow T_{n-1}(A)$$
	defined for each short exact sequence $0\rightarrow A \rightarrow B \rightarrow C \rightarrow 0$ in $\mathcal{A}$.
	We will assume that $T_n = 0$ for $n<0$. \\
	The following two conditions are imposed:
	\begin{enumerate}[label=\arabic*.]
		\item For each short exact sequence $0 \rightarrow A \rightarrow B \rightarrow C \rightarrow 0$, there is a long exact sequence
		$$\dots T_{n+1}(C) \overset{\delta}{\rightarrow} T_n(A) \rightarrow T_n(B) \rightarrow T_n(C) \overset{\delta}{\rightarrow} T_{n-1}(A) \dots$$
		In particular, $T_0$ is right exact, and $T^0$ is left exact.
		
		\item For each morphism of short exact sequences from $0 \rightarrow A' \rightarrow B' \rightarrow C' \rightarrow 0$ to $0 \rightarrow A \rightarrow B \rightarrow C \rightarrow 0$, the $\delta$'s give a commutative diagram
		\begin{align*}
			\begin{gathered}
				\xymatrixcolsep{5pc}\xymatrixcolsep{2pc}\xymatrix{
					T_n(C') \ar[r]^{\delta} \ar[d] & T_{n-1}(A') \ar[d] \\
					T_n(C) \ar[r]^{\delta} & T_{n-1}(A). }
			\end{gathered}
		\end{align*}
	\end{enumerate}
\end{definition}

\begin{example}
	Homology gives a homological $\delta$-functor $H_*$ from $\ch_{\geq0}\cat{A}$ to $\cat{A}$.
\end{example}

\begin{exercise}
	Let $\mathcal{S}$ be the category of short exact sequences
	\begin{equation}
		0 \rightarrow A \rightarrow B \rightarrow C \rightarrow 0 \tag{\textasteriskcentered}
	\end{equation}
	in $\mathcal{A}$. \\
	Then $\delta_i$ is a natural transformation from the functor $F$ sending (\textasteriskcentered) to $T_i(C)$ to the functor $G$ sending (\textasteriskcentered) to $T_{i-1}(A)$.
\end{exercise}

\begin{proof}
	Let $f:\mathcal{S}\rightarrow\mathcal{S}$ be a morphism of short exact sequences, and let $0 \rightarrow A' \rightarrow B' \rightarrow C' \rightarrow 0$ (\textasteriskcentered') be the image of $0 \rightarrow A \rightarrow B \rightarrow C \rightarrow 0$ (\textasteriskcentered) under $f$. \\
	By definition, we know that
	\begin{align*}
		\begin{CD}
			F(\textasteriskcentered) @= T_i(C) @>\delta>> T_{i-1}(A) @= G(\textasteriskcentered) \\
			@.  				@VVF(f)V 			@VG(f)VV 			@. \\
			F(\textasteriskcentered') @= T_i(C') @>\delta>> T_{i-1}(A') @= G(\textasteriskcentered').
		\end{CD}
	\end{align*}
	commutes.
	Therefore, $\delta_i$ is a natural transformation.
\end{proof}

\begin{example}[p-torsion]
	If $p$ is an integer, the functors $T_0(A)=A/pA$ and 
	$$T_1(A) = {_p}A \equiv \{a\in A:pa=0\}$$
	fit together to form a homological $\delta$-functor, or a cohomological $\delta$-functor (with $T^0=T_1$ and $T^1=T_0$) from \textbf{Ab} to \textbf{Ab}.\\
	We can apply the \hyperref[snake_lemma]{Snake Lemma} to the commutative diagram
	\begin{align*}
		\begin{CD}
			0 @>>> A @>>> B @>>> C @>>> 0 \\
			@.	 @VpVV	 @VpVV	@VpVV  @. \\
			0 @>>> A @>>> B @>>> C @>>> 0 \\
		\end{CD}
	\end{align*}
	to get an exact sequence
	$$0 \rightarrow {_p}A \rightarrow {_p}B \rightarrow {_p}C \overset{\delta}{\longrightarrow} A/pA \rightarrow B/pB \rightarrow C/pC \rightarrow 0.$$
\end{example}

\begin{definition}
	A \textit{morphism} $S \rightarrow T$ of $\delta$-functor is a system of natural transformations $S_n \rightarrow T_n$ (resp. $S^n \rightarrow T^n$) that commute with $\delta$, i. e. for every short exact sequence $0 \rightarrow A \rightarrow B \rightarrow C \rightarrow 0$ the diagram
	\begin{align*}
		\begin{CD}
			\dots @>>> S_{n+1}(C) @>\delta_{n+1}>> S_n(A) @>>> S_n(B) @>>> S_n(C) @>\delta_n>> \dots \\
			@.			@VVV						@VVV		@VVV		@VVV				@. \\
			\dots @>>> T_{n+1}(C) @>\delta_{n+1}>> T_n(A) @>>> T_n(B) @>>> T_n(C) @>\delta_n>> \dots			
		\end{CD}
	\end{align*}
	commutes. \\
	A homological $\delta$-functor $T$ is \textit{universal} if, given any other $\delta$-functor $S$ and a natural transformation $f_0:S_0\rightarrow T_0$, there exists a unique morphism $\{f_n:S_n\rightarrow T_n\}$ of $\delta$-functors extending $f_0$. \\
	A cohomological $\delta$-functor $T$ is \textit{universal} if, given any other $\delta$-functor $S$ and a natural transformation $f^0:S^0\rightarrow T^0$, there exists a unique morphism $T\rightarrow S$ of $\delta$-functors extending $f_0$.
\end{definition}

\begin{example}
	We will see that homology $H_*:\ch_{\geq0}(\cat{A}) \rightarrow \cat{A}$ and cohomology $H^*:\ch_{\geq0}(\cat{A})$ are universal $\delta$-functors.
\end{example}

\begin{exercise}
	If $F:\cat{A}\rightarrow\cat{B}$ is an exact functor, then $T_0=F$ and $T_n=0$ for $n\neq0$ defines a universal $\delta$-functor (of both homological and cohomological type).
\end{exercise}

\begin{proof}
	Let $0 \rightarrow A \rightarrow B \rightarrow C \rightarrow 0$ be a short exact sequence in $\cat{A}$. \\
	Consider any morphism $$\delta_n:T_n(C)\rightarrow T_{n-1}(A).$$
	Since $T_n(C)=0$ for $n\neq0$, we have a long exact sequence
	\begin{align*}
		\begin{CD}
			\dots @>>> 0 @>\delta_1>> T_0(A) @>>> T_0(B) @>>> T_0(C) @>\delta_0>> 0 @>>> \dots
		\end{CD}
	\end{align*}
	Let $f$ be a morphism of short exact sequences that maps $0 \rightarrow A \rightarrow B \rightarrow C \rightarrow 0$ to $0 \rightarrow A' \rightarrow B' \rightarrow C' \rightarrow 0$. Consider the following diagram:
	\begin{align*}
		\begin{CD}
			T_n(C) @>\delta_n>> T_{n-1}(A) \\
			@VVV		@VVV \\
			T_n(C') @>\delta_n>> T_{n-1}(A').
		\end{CD}
	\end{align*}
	This diagram commutes for all $n\in\Z$, since all $\delta_n$ are zero maps. \\
	Now we know that the $T_n$'s define a homological $\delta$-functor, and we will show that this is a universal $\delta$-functor. \\
	Let $S$ be another $\delta$-functor and let $f_0:S_0\rightarrow T_0$ be a natural transformation.
	For $n\neq1$, the only possible natural transformation is $f_n:S_n\rightarrow T_n$, since $T_n$ is the trivial functor. \\
	Therefore, $S$ is uniquely defined by $f_0$, and $T$ is a universal $\delta$-functor.
\end{proof}

\subsection{Projective Resolutions}

\begin{definition}
	An object $P$ in an abelian category is called \textit{projective}, if it satisfies the following universal lifting property:
	Given a surjection $g:B\rightarrow C$ and a map $\gamma:P\rightarrow C$, there is at least one map $\beta: P\rightarrow B$ such that $\gamma=g\circ\beta$.
	\begin{align*}
		\xymatrix{
			&P \ar[ld]_{\exists\beta} \ar[d]^\gamma \\
			B \ar[r]_g &C \ar[r] &0
		}
	\end{align*}
\end{definition}

\begin{remark}
	\begin{enumerate}[label=(\roman*)]
		\item Free $R$-modules are projective. \\
		Let $M$ be a free $R$-module with basis $\{x_i\vert i\in I\}$, $g:B\rightarrow C$ a surjection, $\gamma:M\rightarrow C$ a map.
		Since $g$ is surjective, we can find $b_i\in B$ such that $g(b_i)=\gamma(x_i)\in C$ for all $i \in I$.
		Then $M$ is projective with the map $\beta:M\rightarrow B$, $x_i\mapsto b_i$.\\
		\item If a direct sum of $R$-modules is projective, then so are its summands.\\
		Let $\bigoplus_{i\in I}M_i$ be projective, $g:B\rightarrow C$ a surjection, $\gamma_j:M_j \rightarrow C$ a map for some $j\in I$.
		Let $\iota_j:M_j\rightarrow \bigoplus_{i\in I}$ be the natural embedding. \\
		Consider the projection $\pi_j:\bigoplus_{i\in I}M_i\rightarrow M_j$. For $\gamma_j\circ\pi_j$, there is a lift $\beta_j:\bigoplus_{i\in I}M_i\rightarrow B$.
		\begin{align*}
			\xymatrix{
				\bigoplus_{i\in I}M_i \ar[r]^{\pi_j} \ar[d]^{\beta_j} \ar@{.>}[rd]^{\gamma_j\circ\pi_j} &M_j \ar[d]^{\gamma_j} \ar@/_2pc/[l]^{\iota_j} \\
				B \ar[r] &C \ar[r] &0
			}
		\end{align*}
		 Then $\beta_j\circ\iota_j: M_j\rightarrow B$ is a lift for $\gamma_j$, since for all $x \in M_j$:
		 $$\gamma_j(x) = (\gamma_j \circ (\pi_j \circ \iota_j))(x) = ((\gamma_j \circ \pi_j) \circ \iota_j)(x) = (\beta_j \circ \iota_j)(x).$$
	\end{enumerate}
\end{remark}

\begin{proposition}
	An $R$-module is projective if and only if it is a direct summand of a free $R$-module.
\end{proposition}

\begin{proof}
	Let $A$ be the a projective $R$-module. Let $R\{A\}$ denote the free $R$-module generated by the set of $A$. \\
	We have a surjection $R\{A\}\overset{\pi}{\longrightarrow}A$. Since $A$ is projective, we have a map $\iota:A\rightarrow R\{A\}$ such that $id_A=\pi\circ\iota$. \\
	We can restrict $\iota$ to the kernel of $\pi$ to get a short exact sequence
	$$0\rightarrow \ker A \rightarrow R\{A\} \rightarrow A \rightarrow 0.$$
	We can also restrict $\pi$ to the map $\pi':R\{\ker A\}\rightarrow \ker A$ and we have $\pi\circ\iota\vert_{\ker A}(x) = \pi'\circ\iota\vert_{\ker A}(x) = \id_{\ker A}(x)$ for all $x\in\ker A$. \\
	Therefore, the short exact sequence splits and we get an isomorphism $$R\{A\}\cong \ker A \oplus A.$$
	For the other implication, let $A$ be an direct summand of a free $R$-module $F$. Then $F$ is projective and so is $A$. 
\end{proof}

\begin{example}
	There are a lot of nice rings where every projective module is also free, e. g. $\Z$, fields and division rings.
	But this is not always the case.
	\begin{enumerate}[label=(\roman*)]
		\item Consider $R=R_1\times R_2$. Then $P=R_1\times 0$ and $0\times R_2$ are projective as they are direct summands of $R$. \\
		But $P$ is not free: \\
		Let $\varphi:R_1\times R_2 \overset{\cong}{\longrightarrow} R$. Consider $(0,1)\in R_1\times R_2$.
		Since $\varphi$ is an isomorphism, we know that $\varphi(0,1)\neq0$.
	\end{enumerate}
\end{example}

\begin{remark}
	The category $\cat{A}$ of finite abelian groups is an abelian category with no projective objects. \\
	We say that $\cat{A}$ \textit{has enough projectives} if for every object $A$ of $\cat{A}$ there is a surjection $P\rightarrow A$ with $P$ projective.
\end{remark}

\begin{lemma}
	$M$ is projective iff $\Hom_{\cat{A}}(M,-)$ is an exact functor, i. e. iff the sequence 
	$$0 \rightarrow \Hom(M,A) \rightarrow \Hom(M,B) \rightarrow \Hom(M,C) \rightarrow 0$$
	is exact for every exact sequence $0\rightarrow A \rightarrow B \rightarrow C \rightarrow 0$ in $\cat{A}$.
\end{lemma}

\begin{proof}
	
\end{proof}

\begin{exercise}
	A chain complex $P$ is a projective object in $\ch$ iff it is a split exact complex of projectives.
\end{exercise}

\begin{proof}
	content...
\end{proof}

\begin{exercise}
	If $\cat{A}$ has enough projectives, then so does the category $\ch(\cat{A})$ of chain complexes over $\cat{A}$.
\end{exercise}

\begin{proof}
	content...
\end{proof}

\begin{definition}
	Let $M$ be an object of $\cat{A}$. A \textit{left resolution} of $M$ is a complex $P$ with $P_i=0$ for $i<0$, together with a map $\epsilon:P_0\rightarrow M$ such that the augmented complex
	$$\dots \overset{d}{\longrightarrow} P_2 \overset{d}{\longrightarrow} P_1 \overset{d}{\longrightarrow} P_0 \overset{\epsilon}{\longrightarrow} M \longrightarrow 0$$
	is exact. It is a \textit{projective resolution} if each $P_i$ is projective.
\end{definition}

\begin{lemma}
	Every $R$-module $M$ has a projective resolution. More generally, if an abelian category $\cat{A}$ has enough projectives, then every object $M$ in $\cat{A}$ has a projective resolution.
\end{lemma}

\begin{exercise}
	If $P$ is a complex of projectives with $P_i=0$ for $i<0$, then a map $\epsilon:P_0\rightarrow M$ giving a resolution for $M$ is the same as a chain map $\epsilon:P\rightarrow M$, where $M$ is considered as a complex concentrated in degree zero.
\end{exercise}

\begin{theorem}[Comparison Theorem]
	Let $P\overset{\epsilon}{\rightarrow} M$ be a projective resolution of $M$ and $f':M\rightarrow N$ a map in $\cat{A}$. Then for every resolution $Q\overset{\eta}{\rightarrow} N$ of $N$ there is a chain map $f:P\rightarrow Q$ lifting $f'$ in the sense that $\eta\circ f_0=f'\circ\epsilon$. The chain map $f$ is unique up to chain homotopy equivalence.
	\begin{align*}
		\xymatrix{
			\dots \ar[r] &P_2 \ar[r] \ar[d]^{\exists} &P_1 \ar[r] \ar[d]^{\exists} &P_0 \ar[r]^{\epsilon} \ar[d]^{\exists} &M \ar[r] \ar[d]^{f'} &0 \\
			\dots \ar[r] &Q_2 \ar[r] &Q_1 \ar[r] &Q_0 \ar[r]^{\eta} &N \ar[r] &0 
		}
	\end{align*}
\end{theorem}

\begin{porism}
	In the proof we will see that the hypothesis that $P\rightarrow M$ is a projective resolution is too strong. It suffices to be given a chain complex
	$$\dots \rightarrow P_2 \rightarrow P_1 \rightarrow P_0 \rightarrow M \rightarrow 0$$
	with the $P_i$ projective. Then for every resolution $Q\rightarrow N$ of $N$, every map $M\rightarrow N$ lifts to a map $P\rightarrow Q$, which is unique up to chain homotopy.
	This stronger version of the Comparison Theorem will be used in section 2.7 to construct the external product for $\tor$.
\end{porism}

\begin{proof}
	content...
\end{proof}

\begin{lemma}[Horseshoe Lemma]
	Suppose given a commutative diagram
	\begin{align*}
		\xymatrix{
			& & & &0 \ar[d] & \\
			\dots \ar[r] &P'_2 \ar[r] &P'_1 \ar[r] &P'_0 \ar[r]^{\epsilon'} &A' \ar[r] \ar[d]^{\iota_A} &0 \\
			& & & &A \ar[d]^{\pi_A} & \\
			\dots \ar[r] &P''_2 \ar[r] &P''_1 \ar[r] &P''_0 \ar[r]^{\epsilon''} &A'' \ar[r] \ar[d] &0 \\
			& & & &0 & 
		}
	\end{align*}
	where the column is exact and the rows are projective resolutions. Set $P_n:=P'_n\oplus P''_n$. Then the $P_n$ assemble to form a projective resolution $P$ of $A$ and the right hand column lifts to an exact sequence of complexes
	$$0\rightarrow P' \overset{\iota}{\rightarrow} P \overset{\pi}{\rightarrow} P'' \rightarrow 0,$$
	where $\iota_n:P'_n\rightarrow P_n$ and $\pi_n:P_n\rightarrow P''_n$ are the natural inclusion and projection, respectively.
\end{lemma}

\begin{proof}
	content...
\end{proof}

\begin{exercise}
	There are maps $\lambda_n: P''_n\rightarrow P*_{n-1}$ such that
	$$d=\begin{pmatrix}
			d' & \lambda \\
			0 & d'' \\
	\end{pmatrix},
	\quad\text{i. e.}\quad
	d'\begin{pmatrix}p'\\p''\end{pmatrix}=
		\begin{pmatrix}
			d'(p')+\lambda(p'') \\
			d''(p'')\\
		\end{pmatrix}$$
\end{exercise}

\subsection{Injective Resolutions}

\begin{definition}
	An object $I$ in an abelian category $\cat{A}$ is \textit{injective} if it satisfies the following universal lifting property: \\
	Given an injection $f: A \rightarrow B$ and a map $\alpha:A\rightarrow I$, there exists at least one map $\beta:B\rightarrow I$ such that $\alpha=\beta\circ f$.
	\begin{align*}
		\xymatrix{
			0 \ar[r] &A \ar[r]^f \ar[d]^\alpha & B \ar[ld]^{\exists\beta} \\
			&I
		}
	\end{align*}
	We say that $\cat{A}$ \textit{has enough injectives} if for every object $A$ in $\cat{A}$ there is an injection $A\rightarrow I$ with $I$ injective. 
\end{definition}

\begin{note}
	If $\{I_\alpha\}$ is a family of injectives, then the product $\prod I_\alpha$ is also injective.
	The notion of injective module was invented by R. Baer in 1940, long before projective modules were thought of.
\end{note}

\begin{criterion}[Baer] \label{baer}
	A right $R$-module $E$ is injective if and only if for every right ideal $J$ of $R$, every map $J\rightarrow E$ can be extended to a map $R\rightarrow E$.
\end{criterion}

\begin{proof}
	If $E$ is injective and $J$ a right ideal of $R$, then every map $J \rightarrow E$ can be lifted to a map $R \rightarrow E$.
	
	Conversely, let $B$ be an $R$-module and let $A$ be a submodule of $B$.
	Let $\alpha: A \rightarrow E$ be a map.
	We need to extend $\alpha$ to a map $\alpha'': B \rightarrow E$.
	Let $\mathcal{E}$ be the set of all extensions of $\alpha$, partially ordered by $(\alpha_1, A_1) \leq (\alpha_2, A_2) \iff A_1 \subseteq A_2$ and $\alpha_2\vert_{A_1} = \alpha_1$.
	Zorn's lemma gives us a maximal extension $\alpha': A' \rightarrow E$.
	Suppose there is a $b \in B \setminus A'$.
	Then $J := \{r \in R \vert br \in A'\}$ is a right ideal of $R$.
	By assumption, we can extend the composition $J \overset{b}{\longrightarrow} A' \overset{\alpha'}{\longrightarrow} E$ to a map $f: R \rightarrow E$.
	Let $A'' = A' + bR$ be a submodule of $B$ and define $\alpha'': A'' \rightarrow E$ by
	\[\alpha''(a + br) := \alpha'(a) + f(r),\]
	where $a \in A'$ and $r \in R$.
	We need to show that this is a well-defined map.
	Assume that $a_0 + br_0 = a_1 + br_1$ with $a_0, a_1 \in A'$ and $r_0, r_1 \in R$.
	Then $b(r_0 - r_1) = a_1 - a_0 \in A'$, hence $r_0 - r_1 \in J$.
	Then we can see that
	\[\alpha'(a_1) - \alpha'(a_0) = \alpha'(a_1 - a_0) = \alpha'(b(r_0 - r_1)) = f(r_0 - r_1) = f(r_0) - f(r_1).\]
	Therefore, $\alpha'(a_1) + f(r_1) = \alpha'(a_0) + f(r_0)$, hence $\alpha''$ is well-defined.
	Let $a \in A'$.
	Then $\alpha''(a) = \alpha'(a)$, hence $\alpha''\vert_{A'} = \alpha_1$.
	Since we assumed that $A' \subsetneq B$, this contradicts the maximality of $\alpha'$.
	Therefore, $A' = B$ and we can conclude that $E$ is injective.
\end{proof}

\begin{exercise}
	Let $R=\Z/m$. Then $R$ is an injective $R$-module.
	Furthermore, $\Z/d$ is not an injective $R$-module when $d\vert m$ and some prime divides both $d$ and $m/d$
\end{exercise}

\begin{proof}
	We need to show that $\varphi: I \rightarrow \Z/m\Z$ can be extended to $\bar{\varphi}: \Z/m\Z \rightarrow \Z/m\Z$ for every ideal $I$ of $\Z/m\Z$.
	Let $I$ be an ideal of $\Z/m\Z$.
	Then $I$ is of the form $d\Z/m\Z \cong \Z/(\frac{m}{d})\Z$ for some $d \vert m$.
	Since \[\frac{m}{d}f(1) = f(\frac{m}{d}) = f(0) = 0,\]
	we know that $f(1) \in d\Z$, otherwise $\varphi$ would not be a well-defined morphism.
	Now we can extend $f$ to a morphism $\bar{f}: \Z/m\Z \rightarrow \Z/m\Z$ via $1 \mapsto \frac{f(1)}{d}$.
	This is well-defined, since $d \vert f(1)$.
	By Baer's Criterion, $\Z/m\Z$ is an injective $\Z/m\Z$-module.
	
	Let $d\vert m$ such that there is a prime with $p \vert d$ and $p \vert \frac{m}{d}$.
	Let $f: \Z/p \rightarrow \Z/m$ be the map $1 \mapsto \frac{m}{p}$.
	Let $g: \Z/p \rightarrow \Z/d$ be the map $1 \mapsto \frac{d}{p}$.
	Since $\ker(f) = \{x \in \Z/p : p \vert x\} = \{0\}$ and $\ker(g) = \{x \in \Z/p : p \vert x\} = \{0\}$, we know that $f$ and $g$ are injective.
	Now consider any map $h: \Z/m \rightarrow \Z/d$.
	Then any multiple of $d$ in $\Z/m$ gets send to $0$, since $h(d\cdot a) = d \cdot h(a) = 0$ in $\Z/d$.
	We also know that $p \vert \frac{m}{d}$.
	Then $d \vert \frac{m}{p}$, and therefore $h\circ f= 0$, which proves that $h\circ f \neq g$ for all $h$.
	Therefore $\Z/d$ cannot be an injective.
\end{proof}

\begin{corollary}
	Suppose that $R=\Z$, or more generally that $R$ is a principle ideal domain. An $R$-module $A$ is injective iff it is divisible, i. e., for every $r\neq0$ in $R$ and every $a\in A, a=br$ for some $b\in A$.
\end{corollary}

\begin{proof}
	Let $R$ be a principle ideal domain and let $A$ be an $R$-module.
	
	Suppose that $A$ is injective.
	Let $r \in R\setminus\{0\}$ an let $a \in A$.
	Consider the homomorphism $f: R \rightarrow R$, $1 \mapsto r$.
	Since $r\neq0$ and $R$ is a principal ideal domain, we know that $f(x) = rx  = 0 \iff x = 0$.
	Hence, $f$ is injective.
	Now consider the map $\alpha: R \rightarrow A$, $1 \mapsto a$.
	We can lift $\alpha$ to the map $\beta: R \rightarrow A$, i. e. $\alpha = \beta\circ f$.
	We then know
	\[a = \alpha(1) = \beta(f(1)) = \beta(r) = r \cdot \beta(1).\]
	
	Conversely, let $J$ be any ideal of $R$ and let $f: J \rightarrow A$ be any map.
	Since $R$ is a principle ideal domain, there is a generator $r \in R$ of $J$.
	If $r=0$, the zero map $R \rightarrow A$ extends the map $0 = J \rightarrow A$.
	If $r \neq 0$, then there is a $b \in A$ such that $f(b) = br$, since $A$ is divisible.
	Consider $\beta: R \rightarrow A$, $1 \mapsto b$.
	Then \[\beta(rk) = b(rk) = (br)k = f(r)k = f(rk),\] hence $\beta$ extends $f$.
	By \hyperref[baer]{Baer's Criterion}, $A$ is injective.
\end{proof}

\begin{example}
	The divisible abelian groups $\Q$ and $\Z_{p^\infty}=\Z[\frac{1}{p}]/\Z$ are injective ($\Z[\frac{1}{p}]$ is the group of rational numbers of the form $a/p^n,n\geq 1$). Every injective abelian group is a direct sum of these. In particular, the injective abelian group $\Q/\Z$ is isomorphic to $\bigoplus\Z_{p^\infty}$.
\end{example}

The category $\ab$ has enough injectives.
Consider an abelian group $A$. Let $I(A)$ be the product of copies of the injective group $\Q/\Z$, indexed by $\Hom_{\ab}(A,\Q/\Z)$.
Then $I(A)$ is injective, being a product of injectives, and there is a canonical map $e_A:A\rightarrow I(A)$.
This is the desired injection of $A$ into an injective.

\begin{exercise}
	$e_A$ is an injection.
\end{exercise}

\begin{proof}
	content...
\end{proof}

\begin{exercise}
	An abelian group $A$ is zero iff $\Hom_{\ab}(A,\Q/\Z)=0$.
\end{exercise}

\begin{proof}
	content...
\end{proof}

\begin{remark}
	It is easily verified, that if $\cat{A}$ is an abelian category, then $\cat{A}^{op}$ is also abelian.
	The definition of injective is dual to the definition of projective, so the following result are easily deduced by arguing in $\cat{A}^{op}$.
\end{remark}

\begin{proof}
	Let $\cat{A}$ be an abelian category.
	For every diagram
	\[
		\xymatrix{
			A \ar[r]^f &B \ar@<-.5ex>[r]_g \ar@<.5ex>[r]^{g'} &C \ar[r]^h &D
		}
	\]
	in $\cat{A}^{op}$ we have $f \in \Hom_{\cat{A}}(B, A)$, $g,g' \in \Hom_{\cat{A}}(C, B)$ and $h \in \Hom_{\cat{A}}(D,C)$.
	Since $\cat{A}$ is an $\ab$-category, we know that $f(g+g')h = fgh + fg'h$ in $\cat{A}$.
	Therefore, we know $h(g+g')h = hgf + hg'f$ in $\cat{A}^{op}$, and $\cat{A}^{op}$ is an $\ab$-category.
	
	We know that $\Hom_{\cat{A}^{op}}(0,X) = \Hom_{\cat{A}}(X,0)$ for all $X \in \Ob(\cat{A}^{op})$ and $\Hom_{\cat{A}^{op}}(X,0) = \Hom_{\cat{A}}(0,X)$ for all $X \in \Ob(\cat{A}^{op})$.
	Therefore, $0$ is a zero object in $\cat{A}^{op}$.
	Let $B,C$ be two objects of $\cat{A}^{op}$.
	Since $\cat{A}$ is an additive category, the coproduct $B\coprod C$ exists in $\cat{A}$ and coincides with the product $B\times C$.
	By duality, the product and coproduct of $B$ and $C$ exist in $\cat{A}^{op}$ and they coincide.
	Therefore, $\cat{A}^{op}$ is an additive category.
	
	Let $f \in \Hom_{\cat{A}^{op}}(B,C)$ be a morphism.
	Then $f$ is a morphism in $\Hom_{\cat{A}}(C,B)$ and has a kernel and cokernel in $\cat{A}$.
	By duality, $f$ has a cokernel and kernel in $\cat{A}^{op}$.
	Let $f$ be a monomorphism in $\cat{A}^{op}$.
	Then $f$ is an epimorphism in $\cat{A}$ is the cokernel of its kernel.
	By duality, $f$ is the kernel of its cokernel in $\cat{A}^{op}$.
	Let $f$ be an epimorphism in $\cat{A}^{op}$.
	Then $f$ is a monomorphism in $\cat{A}$ is the kernel of its cokernel.
	By duality, $f$ is the cokernel of its kernel in $\cat{A}^{op}$.
	Therefore, $\cat{A}^{op}$ is an abelian category.
\end{proof}

\begin{lemma}
	Let $I$ be an object in an abelian category $\cat{A}$. The following statements are equivalent:
	\begin{enumerate}[label=(\roman*)]
		\item $I$ is injective in $\cat{A}$.
		
		\item $I$ is projective in $\cat{A}^{op}$.
		
		\item The contravariant functor $\Hom_\cat{A}(-,I)$ is exact, i. e. it takes short exact sequences in $\cat{A}$ to short exact sequences in $\ab$.
	\end{enumerate}
\end{lemma}

\begin{proof}
	Let $I$ be an injective object in $\cat{A}$.
	It is easy to see, that $I$ must be a projective object in $\cat{A}^{op}$.
	\[
		\vcenter{\xymatrix{
			0 \ar[r] &A \ar[r]^f \ar[d]^\alpha & B \ar[ld]^{\exists\beta} \\
			&I
		}}
		\quad\text{in $\cat{A}$ becomes}\quad
		\vcenter{\xymatrix{
			0  &A \ar[l]   & B \ar[l]^f  \\
			&I \ar[u]^\alpha \ar[ru]_{\exists\beta}
		}}
	\quad\text{in $\cat{A}^{op}$}
	\]
	Similarly, a projective object in $\cat{A}^{op}$ is an injective object in $(\cat{A}^{op})^{op} = \cat{A}^{op}$.
	
	The third statement can also be viewed in $\cat{A}^{op}$.
	It is equivalent to the statement
\end{proof}

\begin{definition}
	Let $M$ be an object of $\cat{A}$.
	A \textit{right resolution} of $M$ is a cochain complex $I$ with $I^i=0$ for $i<0$ and a map $M\rightarrow I^0$ such that the augmented complex
	$$0 \rightarrow M \rightarrow I^0 \overset{d}{\rightarrow} I^1 \overset{d}{\rightarrow} I^2 \overset{d}{\rightarrow} \dots$$
	is exact.
	This is the same as a cochain map $M\rightarrow I$, where $M$ is considered as a complex concentrated in degree $0$.
	It is called an \textit{injective resolution} if each $I^i$ is injective.
\end{definition}

\begin{lemma}
	If the abelian category $\cat{A}$ has enough injectives, then every object in $\cat{A}$ has an injective resolution.
\end{lemma}

\begin{proof}
	content...
\end{proof}

\begin{theorem}[Comparison Theorem]
	Let $N\rightarrow I$ be an injective resolution of $N$ and $f':M\rightarrow N$ a map in $\cat{A}$. Then for every resolution $M\rightarrow E$ there is a cochain map $F:E\rightarrow I$ lifting $f'$.
	The map $f$ is unique up to cochain homotopy equivalence.
	\begin{align*}
		\xymatrix{
			0 \ar[r] &M \ar[r] \ar[d]^{f'} &E^0 \ar[r] \ar[d]^{\exists} &E^1 \ar[r] \ar[d]^{\exists} &E^2 \ar[r] \ar[d]^{\exists} &\dots \\
			0 \ar[r] &N \ar[r] &I^0 \ar[r] &I^1 \ar[r]^{\eta} &I^2 \ar[r] &\dots
		}
	\end{align*}
\end{theorem}

\begin{exercise}
	$I$ is an injective object in the category of chain complexes iff $I$ is a split exact complex of injectives. \\
	If $\cat{A}$ has enough injectives, then $\ch(\cat{A})$ of chain complexes over $\cat{A}$.
\end{exercise}

\begin{lemma}
	For every right $R$-module $M$, the natural map
	$$\tau:\Hom_{\ab}(M,A)\rightarrow\Hom_{mod-R}(M,\Hom_{\ab}(R,A))$$
	is an isomorphism, where $(\tau f)(m)$ is the map $r\mapsto f(mr)$.
\end{lemma}

\begin{proof}
	content...
\end{proof}

\begin{definition}
	A pair of functors $L:\cat{A}\rightarrow\cat{B}$ and $R:\cat{B}\rightarrow\cat{A}$ are \textit{adjoint} if there is a natural bijection for all $A$ in $\cat{A}$ and $B$ in $\cat{B}$:
	$$\tau = \tau_{AB}:\Hom_{\cat{B}}(L(A),B) \overset{\cong}{\rightarrow} \Hom_{\cat{A}}(A,R(B)).$$
	Here, "natural" means that for all $f:A\rightarrow A'$ in $\cat{A}$ and $g:B\rightarrow B'$ in $\cat{B}$ the following diagram commutes:
	\begin{align*}
		\xymatrix{
			\Hom_{\cat{B}}(L(A'),B) \ar[r]^{Lf^*} \ar[d]^{\tau} &\Hom_{\cat{B}}(L(A),B) \ar[r]^{g_*} \ar[d]^{\tau} &\Hom_{\cat{B}}(L(A),B') \ar[d]^{\tau} \\
			\Hom_{\cat{A}}(A',R(B)) \ar[r]^{f^*} &\Hom_{\cat{A}}(A,R(B)) \ar[r]^{Rg_*} &\Hom_{\cat{A}}(A,R(B')).
		}
	\end{align*}
	We call $L$ the \textit{left adjoint} and $R$ the \textit{right adjoint} of this pair.
	The above lemma states that the forgetful functor from $mod-R$ to $\ab$ has $\Hom_{\ab}(R,-)$ as its right adjoint.
\end{definition}

\begin{proposition}\label{PresInj}
	If an additive functor $R:\cat{B}\rightarrow\cat{A}$ is right adjoint to an exact functor $L:\cat{A}\rightarrow\cat{B}$ and $I$ is an injective object of $\cat{B}$, then $R(I)$ is an injective object of $\cat{A}$. (We say that $R$ preserves injectives.)
	
	Dually, if an additive functor $L:\cat{A}\rightarrow\cat{B}$ is left adjoint to an exact functor $R:\cat{B}\rightarrow\cat{A}$ and $P$ is a projective object of $\cat{A}$, then $L(P)$ is a projective object of $\cat{B}$. (We say that $L$ preserves projectives.)
\end{proposition}

\begin{proof}
	content...
\end{proof}

\begin{corollary}
	If $I$ is an injective abelian group, then $\Hom_{\ab}(R,I)$ is an injective $R$-module.
\end{corollary}

\begin{exercise}
	If $M$ is an $R$-module, let $I(M)$ be the product of copies of $I_0=\Hom_{\ab}(R,\Q/\Z)$, indexed by the set $\Hom_R(M,I_0)$.
	There is a canonical map $e_M:M\rightarrow I(M)$, which is an injection.
	Being a product of injectives, $I(M)$ is an injective, so this proves that $R-mod$ has enough injectives.
	An important consequence of this is that every $R$-module has an injective resolution.
\end{exercise}

\begin{proof}
	content...
\end{proof}

\begin{example}
	The category $\sheaves(X)$ of abelian group sheaves on a topological space $X$ has enough injectives.
	To see this, we need two constructions.
	The \textit{stalk} of a sheaf $\mathcal{F}$ at a point $x\in X$ is the abelian group $\mathcal{F}_x = \underset{\longrightarrow}{\lim}\{\mathcal{F}(U):x\in U\}$.
	"Stalk at $x$" is an exact functor from $\sheaves(X)$ to $\ab$.
	If $A$ is an abelian group, the \textit{skyscraper sheaf} $x_*A$ at the point $x\in X$ is defined to be the presheaf
	$$(x_*A)(U)=\left\{\begin{array}{ll}A & \text{if }x\in U \\ 0 & \text{otherwise}\end{array}\right.$$
\end{example}

\begin{exercise}
	$x_*A$ is a sheaf and 
	$$\Hom_{\ab}(\mathcal{F}_x,A) \cong \Hom_{\sheaves(X)}(\mathcal{F},x_*A)$$
	for every sheaf $\mathcal{F}$.
	Furthermore, if $A_x$ is an injective abelian group, then $x_*(A_x)$ is an injective object in $\sheaves(X)$ for each $x$, and $\prod_{x\in X} x_*(A_*)$ is also injective.
\end{exercise}

\begin{proof}
	content...
\end{proof}

\begin{example}
	Let $I$ be a small category and $\cat{A}$ an abelian category.
	If the product of any set of objects exists in $\cat{A}$ ($\cat{A}$ is complete) and $\cat{A}$ has enough injectives, we will show that the functor category $\cat{A}^I$ has enough injectives.
	For each $k$ in $I$, the $k^\text{th}$ coordinate $A\mapsto A(k)$ is an exact functor from $\cat{A}^I$ to $A$. Given $A$ in $\cat{A}$, define the functor $k_*A:I\rightarrow\cat{A}$ by sending $i\in I$ to
	$$k_*A(i)=\prod_{\Hom_{I}(i,k)}A.$$
	If $\eta:i\rightarrow j$ is a map in $I$, the map $k_*A(i)\rightarrow k_*A(j)$ is determined by the index map $\eta^*:\Hom(j,k)\rightarrow \Hom(i,k)$.
	That is, the coordinate $k_*A(i)\rightarrow A$ of this map corresponding to $\varphi\in\Hom(j,k)$ is the projection of $k_*A(i)$ onto the factor corresponding to $\eta^*\varphi=\varphi\eta\in\Hom(i,k)$.
	If $f:A\rightarrow B$ is a map $k_*A\rightarrow k_*B$ defined slotwise.
	In this way, $k_*$ becomes an additive functor from $\cat{A}$ to $\cat{A'}$, assuming that $\cat{A}$ has enough products for $k_*A$ to be defined.
\end{example}

\begin{exercise}
	Let $\cat{A}$ be complete and with enough injectives.
	Then $k_*$ is right adjoint to the $k^\text{th}$ coordinate functor, so that $k_*$ preserves injectives by \ref{PresInj}.
	Furthermore, $\mathcal{A}^I$ has enough injectives.
\end{exercise}

\begin{exercise}
	Let $\mathcal{A}$ be cocomplete and have enough projectives.
	Then $\mathcal{A}^I$ has enough projectives.
\end{exercise}

\subsection{Left Derived Functors}

\subsection{Right Derived Functors}

	\printbibliography[
		heading=bibintoc,
		title={Bibliography}
	]
\end{document}


