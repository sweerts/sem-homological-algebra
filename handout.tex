\documentclass[a4paper,11pt]{article}
\usepackage[utf8]{inputenc}


%%%%%%%%%%%%%%%%%%%%%%%%%%%%%%%%%%%%%%%%%%%%%%%%%%%%%%%%%%%%%%%%%%%%%%%%%%
% Hier kommen Einstellungen

% Seitenränder
\usepackage{geometry}
\geometry{
	a4paper,
	left=30mm,
	right=30mm,
	top=35mm,
	bottom=35mm,
}

\usepackage{amssymb, amsfonts, amsbsy, amsmath, amsthm}
\usepackage[english]{babel}
\usepackage{enumitem}

% Bibliographie
\usepackage[german=quotes]{csquotes}
\usepackage[backend=biber, style=alphabetic, url=false, isbn=false, maxbibnames=99, giveninits=true]{biblatex}
\renewbibmacro{in:}{}
\addbibresource{literatur.bib}

% Graphik und kommutative Diagramme
\usepackage{graphicx}
\usepackage{tikz}
\usepackage[all,cmtip]{xy}

% Etwas Farbe, man kann sogar per rgb Wert definieren
\usepackage{color}
\newcommand{\black}{\color{black}}
\newcommand{\red}{\color{red}}
\newcommand{\green}{\color{dgreen}}
\newcommand{\blue}{\color{blue}}
\newcommand{\magenta}{\color{magenta}}
\definecolor{darkblue}{rgb}{0,0,.5}

% Das macht, dass man im pdf auf die Verweise klicken kann
\usepackage[colorlinks,linkcolor=darkblue,citecolor=darkblue,urlcolor=darkblue]{hyperref}

% Fuer Randbemerkungen
\usepackage{marginnote}
\usepackage[color]{changebar}
\changebarsep0.2cm
\changebarwidth0.03cm
\cbcolor{red}


% Ringe und Körper
\newcommand{\N}{{\mathbb N}}
\newcommand{\Z}{{\mathbb Z}}
\newcommand{\Q}{{\mathbb Q}}
\newcommand{\R}{{\mathbb R}}
\newcommand{\C}{{\mathbb C}}

% Potenzmenge
\renewcommand{\P}{{\mathfrak P}}

% Abbildungen
\newcommand{\id}{\operatorname{id}}
\newcommand{\Abb}{\operatorname{Abb}}

% Kategorien
\newcommand{\cat}[2]{\operatorname{#1}(#2)}

% Theorem Definitionen mit durchlaufender Numerierung
%\swapnumbers
\theoremstyle{definition}
\newtheorem{definition}{Definition}[subsection]
%\theoremstyle{plain}
\newtheorem{remark}{Remark}[subsection]
\newtheorem{example}{Example}[subsection]
\newtheorem{proposition}{Proposition}[subsection]
\newtheorem{lemma}{Lemma}[subsection]
\newtheorem{theorem}{Theorem}[subsection]
\newtheorem{algorithm}{Algorithm}[subsection]
\newtheorem{corollary}{Corollary}[subsection]
\newtheorem{assumption}{Assumption}[subsection]
\newtheorem{axiom}{Axiom}[subsection]
\newtheorem{exercise}{Exercise}[subsection]

% Beweise
%\renewcommand{\proofname}{Proof}

\title{Derived Functors}
\author{Steffen Weerts}
\date{\today}

%%%%%%%%%%%%%%%%%%%%%%%%%%%%%%%%%%%%%%%%%%%%%%%%%%%%%%%%%%%%%%%%%%%%%%%%%%
% Jetzt kommt der Text

\begin{document}
	\maketitle
	
	\section{Introduction}
This talk is based on \cite{Wei94}.

\begin{note}[Snake Lemma]\label{snake_lemma}
	Let
	\begin{align*}
		\begin{CD}
			0 @>>> A @>f>> B @>g>> C @>>> 0 \\
			@.   @VaVV   @VbVV    @VcVV    @. \\
			0 @>>> A' @>f'>> B' @>g'>> C' @>>> 0
		\end{CD}
	\end{align*}
	be a commutative diagram with exact rows. \\
	Then there is an exact sequence
	$$0 \rightarrow \ker(a) \rightarrow \ker(b) \rightarrow \ker(c) \rightarrow \coker(a) \rightarrow \coker(b) \rightarrow \coker(c) \rightarrow 0.$$
\end{note}

\begin{definition}[Mapping Cone]
	Let $f:B \rightarrow C$ be a map of chain complexes.
	The \textit{mapping cone} of $f$ is the chain complex whose degree $n$ part is $B_{n-1} \oplus C_n$, i. e.
	\[\dots \rightarrow B_{n} \oplus C_{n+1} \rightarrow B_{n-1} \oplus C_n \rightarrow B_{n-2} \oplus C_{n-1} \rightarrow \dots\]
	The differential is given by
	\[d(b,c) = (-d(b), d(c) - f(c)), \quad (b\in B_{n-1}, c\in C_n).\]
	There is a short exact sequence $0 \rightarrow C \rightarrow \cone(f) \rightarrow B[-1] \rightarrow 0.$ \\
	The mapping cone of $id_C$ (denoted by $\cone(C)$) is \textit{split exact}, i. e.
	\begin{enumerate}[label=(\roman*)]
		\item there are maps $s_n:C_n\rightarrow C_{n+1}$ such that $d=dsd$ ($C$ is \textit{split}),
		\item $C$ is exact as a sequence (\textit{acyclic}).
	\end{enumerate} 
\end{definition}
	\section{Derived Functors}

\subsection{$\delta$-Functors}

\begin{definition}
	A (covariant) homological $\delta$-functor between $\mathcal{A}$ and $\mathcal{B}$ is a collection of additive functors $T_n:\mathcal{A}\rightarrow\mathcal{B}$ for $n\geq0$, together with morphisms
	$$\delta_n:T_n(C)\rightarrow T_{n-1}(A)$$
	defined for each short exact sequence $0\rightarrow A \rightarrow B \rightarrow C \rightarrow 0$ in $\mathcal{A}$.
	We will assume that $T_n = 0$ for $n\leq 0$. \\
	The following two conditions are imposed:
	\begin{enumerate}[label=\arabic*.]
		\item For each short exact sequence $0 \rightarrow A \rightarrow B \rightarrow C \rightarrow 0$, there is a long exact sequence
		$$\dots T_{n+1}(C) \overset{\delta}{\rightarrow} T_n(A) \rightarrow T_n(B) \rightarrow T_n(C) \overset{\delta}{\rightarrow} T_{n-1}(A) \dots$$
		In particular, $T_0$ is right exact, and $T^0$ is left exact.
		
		\item For each morphism of short exact sequences from $0 \rightarrow A' \rightarrow B' \rightarrow C' \rightarrow 0$ to $0 \rightarrow A \rightarrow B \rightarrow C \rightarrow 0$, the $\delta$'s give a commutative diagram
		\begin{align*}
			\begin{gathered}
				\xymatrixcolsep{5pc}\xymatrixcolsep{2pc}\xymatrix{
				T_n(C') \ar[r]^{\delta} \ar[d] & T_{n-1}(A') \ar[d] \\
				T_n(C) \ar[r]^{\delta} & T_{n-1}(A). }
			\end{gathered}
		\end{align*}
	\end{enumerate}
\end{definition}

\begin{example}
	Homology gives a homological $\delta$-functor $H_*$ from $\cat{Ch_{\geq0}}{\mathcal{A}}$ to $\mathcal{A}$.
\end{example}

\begin{exercise}
	Let $\mathcal{S}$ be the category of short exact sequences
	\begin{equation}
		0 \rightarrow A \rightarrow B \rightarrow C \rightarrow 0 \tag{\textasteriskcentered}
	\end{equation}
	in $\mathcal{A}$. \\
	Then $\delta_i$ is a natural transformation from the functor sending (\textasteriskcentered) to $T_i(C)$ to the functor sending (\textasteriskcentered) to $T_{i-1}(A)$.
\end{exercise}

\begin{proof}
	Let 
\end{proof}

\subsection{Projective Resolutions}

\subsection{Injective Resolutions}

\subsection{Left Derived Functors}

\subsection{Right Derived Functors}

	\printbibliography[
		heading=bibintoc,
		title={Bibliography}
	]
\end{document}


