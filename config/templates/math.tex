\usepackage{amssymb, amsfonts, amsbsy, amsmath, amsthm, amscd}

% Graphik und kommutative Diagramme
\usepackage{graphicx}
\usepackage{tikz}
\usepackage[all,cmtip]{xy}

% Ringe und Körper
\newcommand{\N}{{\mathbb N}}
\newcommand{\Z}{{\mathbb Z}}
\newcommand{\Q}{{\mathbb Q}}
\newcommand{\R}{{\mathbb R}}
\newcommand{\C}{{\mathbb C}}

% Potenzmenge
\renewcommand{\P}{{\mathfrak P}}

% Abbildungen
\newcommand{\id}{\operatorname{id}}
\newcommand{\Abb}{\operatorname{Abb}}
%\newcommand{\ker}{\operatorname{ker}}
\newcommand{\coker}{\operatorname{coker}}

% Kategorien
\newcommand{\cat}[1]{\mathcal{#1}}
\newcommand{\ccomplex}[2]{\operatorname{\textbf{Ch}}#1{\geq0}(\mathcal{#2})}
\newcommand{\ch}{\operatorname{\textbf{Ch}}}
\newcommand{\ab}{\operatorname{\textbf{Ch}}}

% Theorem Definitionen mit durchlaufender Numerierung
%\swapnumbers
\theoremstyle{definition}
\newtheorem{theorem}[subsubsection]{Theorem}
\newtheorem{lemma}[subsubsection]{Lemma}
\newtheorem{corollary}[subsubsection]{Corollary}
\newtheorem{proposition}[subsubsection]{Proposition}
\newtheorem{algorithm}[subsubsection]{Algorithm}
\newtheorem{exercise}[subsubsection]{Exercise}

\newtheorem{definition}[subsubsection]{Definition}
\newtheorem{assumption}[subsubsection]{Assumption}
\newtheorem{axiom}[subsubsection]{Axiom}
\newtheorem{notation}[subsubsection]{Notation}

\theoremstyle{plain}
\newtheorem{remark}[subsubsection]{Remark}
\newtheorem{example}[subsubsection]{Example}
\newtheorem{note}[subsubsection]{Note}