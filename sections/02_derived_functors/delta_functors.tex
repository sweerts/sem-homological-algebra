\subsection{$\delta$-Functors}

\begin{notation}
	In this talk, we will use $\mathcal{A}$ and $\mathcal{B}$ as names for two arbitrary abelian categories.
\end{notation}

\begin{definition}
	A (covariant) homological $\delta$-functor between $\mathcal{A}$ and $\mathcal{B}$ is a collection of additive functors $T_n:\mathcal{A}\rightarrow\mathcal{B}$ (resp. $\delta^n:\cat{A}\rightarrow\cat{B}$) for $n\geq0$, together with morphisms
	$$\delta_n:T_n(C)\rightarrow T_{n-1}(A)$$
	$$\text{(resp. } \delta^n:T^n(C)\rightarrow T^{n+1}(A)\text{)}$$
	defined for each short exact sequence $0\rightarrow A \rightarrow B \rightarrow C \rightarrow 0$ in $\mathcal{A}$.
	We will assume that $T_n = T^n = 0$ for $n<0$. \\
	The following two conditions are imposed:
	\begin{enumerate}[label=\arabic*.]
		\item For each short exact sequence $0 \rightarrow A \rightarrow B \rightarrow C \rightarrow 0$, there is a long exact sequence
		$$\dots \rightarrow T_{n+1}(C) \overset{\delta}{\longrightarrow} T_n(A) \rightarrow T_n(B) \rightarrow T_n(C) \overset{\delta}{\longrightarrow} T_{n-1}(A) \dots$$
		(resp.
		$$\dots \rightarrow T^{n-1}(C) \overset{\delta}{\longrightarrow} T^n(A) \rightarrow T^n(B) \rightarrow T^n(C) \overset{\delta}{\longrightarrow} T^{n+1}(A) \dots\text{).}$$
		In particular, $T_0$ is right exact, and $T^0$ is left exact.
		
		\item For each morphism of short exact sequences from $0 \rightarrow A' \rightarrow B' \rightarrow C' \rightarrow 0$ to $0 \rightarrow A \rightarrow B \rightarrow C \rightarrow 0$, the $\delta$'s give a commutative diagram
		\[
			\vcenter{\xymatrix{
				T_n(C') \ar[r]^{\delta} \ar[d] & T_{n-1}(A') \ar[d] \\
				T_n(C) \ar[r]^{\delta} & T_{n-1}(A).
			}}
			\text{ resp. }
			\vcenter{\xymatrix{
				T^n(C') \ar[r]^{\delta} \ar[d] & T^{n+1}(A') \ar[d] \\
				T^n(C) \ar[r]^{\delta} & T^{n+1}(A).
			}}
		\]
	\end{enumerate}
\end{definition}

\begin{example}
	Homology gives a homological $\delta$-functor $H_*$ from $\ch_{\geq0}\cat{A}$ to $\cat{A}$; cohomology gives a cohomological $\delta$-functor $H^*$ from $\ch^{\geq0}(\cat{A})$ to $\cat{a}$.
\end{example}

\begin{exercise}
	Let $\mathcal{S}$ be the category of short exact sequences
	\begin{equation}
		0 \rightarrow A \rightarrow B \rightarrow C \rightarrow 0 \tag{\textasteriskcentered}
	\end{equation}
	in $\mathcal{A}$. \\
	Then $\delta_i$ is a natural transformation from the functor $F$ sending (\textasteriskcentered) to $T_i(C)$ to the functor $G$ sending (\textasteriskcentered) to $T_{i-1}(A)$.
\end{exercise}

\begin{proof}
	Let $f:\mathcal{S}\rightarrow\mathcal{S}$ be a morphism of short exact sequences, and let $0 \rightarrow A' \rightarrow B' \rightarrow C' \rightarrow 0$ (\textasteriskcentered') be the image of $0 \rightarrow A \rightarrow B \rightarrow C \rightarrow 0$ (\textasteriskcentered) under $f$. \\
	By definition, we know that
	\[
		\xymatrix{
			F(\textasteriskcentered) \ar@{=}[r] &T_i(C) \ar[r]^{\delta} \ar[d]^{F(f)} &T_{i-1}(A) \ar@{=}[r] \ar[d]^{G(f)} &G(\textasteriskcentered) \\
			F(\textasteriskcentered') \ar@{=}[r] &T_i(C') \ar[r]^{\delta} &T_{i-1}(A') \ar@{=}[r] &G(\textasteriskcentered').
		}
	\]
	commutes.
	Therefore, $\delta_i$ is a natural transformation.
\end{proof}

\begin{example}[p-torsion]
	If $p$ is an integer, the functors $T_0(A)=A/pA$ and 
	$$T_1(A) = {_p}A \equiv \{a\in A:pa=0\}$$
	fit together to form a homological $\delta$-functor, or a cohomological $\delta$-functor (with $T^0=T_1$ and $T^1=T_0$) from \ab\ to \ab.\\
	We can apply the \hyperref[snake_lemma]{Snake Lemma} to the commutative diagram
	\[
		\xymatrix{
			0 \ar[r] &A \ar[r] \ar[d]^p &B \ar[r] \ar[d]^p &C \ar[r] \ar[d]^p &0 \\
			0 \ar[r] &A \ar[r] &B \ar[r] &C \ar[r] &0
		}
	\]
	to get an exact sequence
	$$0 \rightarrow {_p}A \rightarrow {_p}B \rightarrow {_p}C \overset{\delta}{\longrightarrow} A/pA \rightarrow B/pB \rightarrow C/pC \rightarrow 0.$$
	\textit{Generalization.}
	The same proof shows that if $r$ is any element in a ring $R$, then $T_0(M)=M/rM$ and $T_1(M)={_r}M$ fit together to form a homological $\delta$-functor (or cohomological $\delta$-functor, if that is one's taste) from $R$-\Rmod\ to \ab.
\end{example}

\begin{definition}
	A \textit{morphism} $S \rightarrow T$ of $\delta$-functor is a system of natural transformations $S_n \rightarrow T_n$ (resp. $S^n \rightarrow T^n$) that commute with $\delta$, i. e. for every short exact sequence $0 \rightarrow A \rightarrow B \rightarrow C \rightarrow 0$ the diagram
	\begin{align*}
		\begin{CD}
			\dots @>>> S_{n+1}(C) @>\delta_{n+1}>> S_n(A) @>>> S_n(B) @>>> S_n(C) @>\delta_n>> \dots \\
			@.			@VVV						@VVV		@VVV		@VVV				@. \\
			\dots @>>> T_{n+1}(C) @>\delta_{n+1}>> T_n(A) @>>> T_n(B) @>>> T_n(C) @>\delta_n>> \dots			
		\end{CD}
	\end{align*}
	commutes. \\
	A homological $\delta$-functor $T$ is \textit{universal} if, given any other $\delta$-functor $S$ and a natural transformation $f_0:S_0\rightarrow T_0$, there exists a unique morphism $\{f_n:S_n\rightarrow T_n\}$ of $\delta$-functors extending $f_0$. \\
	A cohomological $\delta$-functor $T$ is \textit{universal} if, given any other $\delta$-functor $S$ and a natural transformation $f^0:S^0\rightarrow T^0$, there exists a unique morphism $T\rightarrow S$ of $\delta$-functors extending $f_0$.
\end{definition}

\begin{example}
	We will see that homology $H_*:\ch_{\geq0}(\cat{A}) \rightarrow \cat{A}$ and cohomology $H^*:\ch_{\geq0}(\cat{A})$ are universal $\delta$-functors.
\end{example}

\begin{exercise}
	If $F:\cat{A}\rightarrow\cat{B}$ is an exact functor, then $T_0=F$ and $T_n=0$ for $n\neq0$ defines a universal $\delta$-functor (of both homological and cohomological type).
\end{exercise}

\begin{proof}
	Let $0 \rightarrow A \rightarrow B \rightarrow C \rightarrow 0$ be a short exact sequence in $\cat{A}$. \\
	Consider any morphism $$\delta_n:T_n(C)\rightarrow T_{n-1}(A).$$
	Since $T_n(C)=0$ for $n\neq0$, we have a long exact sequence
	\begin{align*}
		\begin{CD}
			\dots @>>> 0 @>\delta_1>> T_0(A) @>>> T_0(B) @>>> T_0(C) @>\delta_0>> 0 @>>> \dots
		\end{CD}
	\end{align*}
	Let $f$ be a morphism of short exact sequences that maps $0 \rightarrow A \rightarrow B \rightarrow C \rightarrow 0$ to $0 \rightarrow A' \rightarrow B' \rightarrow C' \rightarrow 0$. Consider the following diagram:
	\begin{align*}
		\begin{CD}
			T_n(C) @>\delta_n>> T_{n-1}(A) \\
			@VVV		@VVV \\
			T_n(C') @>\delta_n>> T_{n-1}(A').
		\end{CD}
	\end{align*}
	This diagram commutes for all $n\in\Z$, since all $\delta_n$ are zero maps. \\
	Now we know that the $T_n$'s define a homological $\delta$-functor, and we will show that this is a universal $\delta$-functor. \\
	Let $S$ be another $\delta$-functor and let $f_0:S_0\rightarrow T_0$ be a natural transformation.
	For $n\neq1$, the only possible natural transformation is $f_n:S_n\rightarrow T_n$, since $T_n$ is the trivial functor. \\
	Therefore, $S$ is uniquely defined by $f_0$, and $T$ is a universal $\delta$-functor.
\end{proof}