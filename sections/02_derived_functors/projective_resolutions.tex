\subsection{Projective Resolutions}

\begin{definition}
	An object $P$ in an abelian category is called \textit{projective}, if it satisfies the following universal lifting property:
	Given a surjection $g:B\rightarrow C$ and a map $\gamma:P\rightarrow C$, there is at least one map $\beta: P\rightarrow B$ such that $\gamma=g\circ\beta$.
	\[
		\xymatrix{
			&P \ar[ld]_{\exists\beta} \ar[d]^\gamma \\
			B \ar[r]_g &C \ar[r] &0
		}
	\]
\end{definition}

\begin{remark}
	\begin{enumerate}[label=(\roman*)]
		\item Free $R$-modules are projective. \\
		Let $M$ be a free $R$-module with basis $\{x_i\vert i\in I\}$, $g:B\rightarrow C$ a surjection, $\gamma:M\rightarrow C$ a map.
		Since $g$ is surjective, we can find $b_i\in B$ such that $g(b_i)=\gamma(x_i)\in C$ for all $i \in I$.
		Then $M$ is projective with the map $\beta:M\rightarrow B$, $x_i\mapsto b_i$.
		
		\item If a direct sum of $R$-modules is projective, then so are its summands.\\
		Let $\bigoplus_{i\in I}M_i$ be projective, $g:B\rightarrow C$ a surjection, $\gamma_j:M_j \rightarrow C$ a map for some $j\in I$.
		Let $\iota_j:M_j\rightarrow \bigoplus_{i\in I}$ be the natural embedding. \\
		Consider the projection $\pi_j:\bigoplus_{i\in I}M_i\rightarrow M_j$. For $\gamma_j\circ\pi_j$, there is a lift $\beta_j:\bigoplus_{i\in I}M_i\rightarrow B$.
		\[
			\xymatrix{
				\bigoplus_{i\in I}M_i \ar[r]^{\pi_j} \ar[d]^{\beta_j} \ar@{.>}[rd]^{\gamma_j\circ\pi_j} &M_j \ar[d]^{\gamma_j} \ar@/_2pc/[l]^{\iota_j} \\
				B \ar[r] &C \ar[r] &0
			}
		\]
		Then $\beta_j\circ\iota_j: M_j\rightarrow B$ is a lift for $\gamma_j$, since for all $x \in M_j$:
		$$\gamma_j(x) = (\gamma_j \circ (\pi_j \circ \iota_j))(x) = ((\gamma_j \circ \pi_j) \circ \iota_j)(x) = (\beta_j \circ \iota_j)(x).$$
	\end{enumerate}
\end{remark}

\begin{proposition}
	An $R$-module is projective if and only if it is a direct summand of a free $R$-module.
\end{proposition}

\begin{proof}
	Let $A$ be the a projective $R$-module. Let $R\{A\}$ denote the free $R$-module generated by the set $A$. \\
	We have a surjection $R\{A\}\overset{\pi}{\longrightarrow}A$.
	Since $A$ is projective, we have a map $\iota:A\rightarrow R\{A\}$ such that $id_A=\pi\circ\iota$. \\
	We can restrict $\iota$ to the kernel of $\pi$ to get a short exact sequence
	$$0\rightarrow \ker A \rightarrow R\{A\} \rightarrow A \rightarrow 0.$$
	We can also restrict $\pi$ to the map $\pi':R\{\ker A\}\rightarrow \ker A$ and we have $\pi\circ\iota\vert_{\ker A}(x) = \pi'\circ\iota\vert_{\ker A}(x) = \id_{\ker A}(x)$ for all $x\in\ker A$. \\
	Therefore, the short exact sequence splits and we get an isomorphism $$R\{A\}\cong \ker A \oplus A.$$
	For the other implication, let $A$ be an direct summand of a free $R$-module $F$. Then $F$ is projective and so is $A$. 
\end{proof}

\begin{example}
	There are a lot of nice rings where every projective module is also free, e. g. $\Z$, fields and division rings.
	But this is not always the case.
	\begin{enumerate}[label=(\roman*)]
		\item Consider the ring $R=\Z/\Z$.
		$R$ is obviously free as an $R$-module.
		Since $R=\Z/2\Z \oplus \Z/3\Z$, we know that $\Z/2\Z$ is a projective $R$-module as direct summand of a free module.
		But $\Z/2\Z$ is not free, since it cannot be isomorphic to a free $R$-module, since these have either $1$ or at least $6$ elements.
		
		\item Consider the ring $R=M_n(F)$ of $n\times n$ matrices over a field $F$, acting on the column vector space $V=F^n$.
		As a left $R$-module, $R$ is the direct sum of its columns, each of which is the left $R$-module $V$.
		Hence $R\cong V\oplus\dots\oplus V$, and $V$ is projective as $R$-module.
		Since any free $R$-module would have dimension $dn^2$ over $F$ for some cardinal number $d$, and $\dim_f(V)=n$, $V$ cannot be free over $R$.
	\end{enumerate}
\end{example}

\begin{remark}
	The category $\cat{A}$ of finite abelian groups is an abelian category with no projective objects. \\
	We say that $\cat{A}$ \textit{has enough projectives} if for every object $A$ of $\cat{A}$ there is a surjection $P\rightarrow A$ with $P$ projective.
\end{remark}

We can characterize projective objects in a different way:

\begin{lemma}
	$M$ is projective iff $\Hom_{\cat{A}}(M,-)$ is an exact functor, i. e. iff the sequence 
	$$0 \rightarrow \Hom(M,A) \rightarrow \Hom(M,B) \rightarrow \Hom(M,C) \rightarrow 0$$
	is exact for every exact sequence $0\rightarrow A \rightarrow B \rightarrow C \rightarrow 0$ in $\cat{A}$.
\end{lemma}

\begin{proof}
	Suppose that $\Hom_{\cat{A}}(M,-)$ is an exact functor.
	Let $g:B\rightarrow C$ be surjective and let $\gamma:M\rightarrow C$ be a map.
	Since $\Hom_{\cat{A}}(M,-)$ is exact, $g_*:\Hom_{\cat{A}}(M,B)\rightarrow\Hom_{\cat{A}}(M,C)$ is surjective.
	Therefore, we can lift $\gamma$ to $\gamma=g_*\beta=\gamma\circ\beta$ and $M$ has the universal lifting property. \\
	Suppose that $M$ is projective.
	Since $\Hom_{\cat{A}}(M,-)$ is a left exact functor, it suffices to show that $g_*$ is surjective for every short exact sequence as above.
	Let $\gamma:M\rightarrow C$ be a map.
	By the universal property, we can lift $\gamma$ to $\beta:M\rightarrow C$ such that $\gamma=g\circ\beta=g_*\beta$, i. e. $g_*$ is surjective.
\end{proof}

A chain complex $P$ in called \textit{chain complex of projectives}, if each $P_n$ is projective.
It is possible that a chain complex of projectives is not a projective object in $\ch$.

\begin{exercise}
	A chain complex $P$ is a projective object in $\ch$ if and only if it is a split exact complex of projectives.
\end{exercise}

\begin{proof}
	Let $P$ be a chain complex that is projective in $\ch$.
	We have a short exact sequence \[0 \rightarrow P \rightarrow \cone(P) \overset{g}{\longrightarrow} P[-1] \rightarrow 0. \tag{$\ast$}\]
	Since $P$ is projective, so is $P[-1]$.
	Since $g$ is surjective, we can lift the identity $\id_{P[-1]}$ to the map $\beta:P[-1]\rightarrow\cone(P)$, i. e. $\id_{P[-1]}=g\circ\beta$.
	Then ($\ast$) is a split exact sequence and we have an isomorphism $\cone(P) \cong P \oplus P[-1]$.
	Since $\cone(P)$ is split exact, we know that
	\[0 = H_n(\cone(P)) \cong H_n(P \oplus P[-1]) \cong H_n(P) \oplus H_n(P[-1]),\]
	Therefore, $H_n(P)=0$ for all $n$ and $P$ is acyclic.
	
	\comment{We need to show that $P$ is split and that each $P_n$ is projective.}
\end{proof}

\begin{exercise}
	If $\cat{A}$ has enough projectives, then so does the category $\ch(\cat{A})$ of chain complexes over $\cat{A}$.
\end{exercise}

\begin{proof}
	Let $\cat{A}$ have enough projectives.
	Let $C$ be a chain complex in $\cat{A}$ with differentials $\delta$.
	Then for every $C_n$ there is a projective $B_n$ such that $B_n \overset{f_n}{\longrightarrow} C_n \rightarrow 0$ is exact.
	Since $B_n$ is projective, we can lift $\delta_n\circ f_n$ to $d^B_n: B_n \rightarrow B_{n-1}$ such that $\delta_n\circ f_n = f_{n-1}\circ d^B_n$. 
	\comment{NEED TO SHOW: $d\circ d =0$.}
	Now define $P_n := B_n \oplus B_{n+1}$ and $d_n: P_n \rightarrow P_{n-1}$ where $d_n(b_n,b_{n+1}) := (d^B_n(b_n), b_n - d^B_{n+1}(b_{n+1}))$.
	This is a chain complex $P$, since $d_n \circ d_{n+1} = 0$:
	\begin{align*}
		d_n( d_{n+1}( b_{n+1}, b_{n+2} ) )
		&= d_n( d'_{n+1}(b_{n+1}), b_{n+1} - d'_{n+2}({b_n+2})) \\
		&= (d'_n(d'_{n+1}(b_{n+1})), d'_{n+1}(b_{n+1}) - d'_{n+1}(b_{n+1} - d'_{n+2}(b_{n+2}))) \\
		&= (0, d'_{n+1}(b_{n+1}) - d'_{n+1}(b_{n+1}) + d'_{n+1}(d'_{n+2}(b_{n+2}))) \\
		&= (0, 0).
	\end{align*}
	
\end{proof}

\begin{definition}
	Let $M$ be an object of $\cat{A}$. A \textit{left resolution} of $M$ is a complex $P$ with $P_i=0$ for $i<0$, together with a map $\epsilon:P_0\rightarrow M$ such that the augmented complex
	$$\dots \overset{d}{\longrightarrow} P_2 \overset{d}{\longrightarrow} P_1 \overset{d}{\longrightarrow} P_0 \overset{\epsilon}{\longrightarrow} M \longrightarrow 0$$
	is exact. It is a \textit{projective resolution} if each $P_i$ is projective.
\end{definition}

\begin{lemma}
	Every $R$-module $M$ has a projective resolution. More generally, if an abelian category $\cat{A}$ has enough projectives, then every object $M$ in $\cat{A}$ has a projective resolution.
\end{lemma}

\begin{proof}
	Let the abelian category $\cat{A}$ have enough projectives and let $M$ be an object of $\cat{A}$.
	Then take a projective object $P_0$ of $\cat{A}$ and a surjection $\epsilon_0:P_0 \rightarrow M$.
	Let $k_0: K_0 \rightarrow P_0$ be the kernel of $\epsilon_0$.
	Inductively, given an object $K_{n-1}$, we can choose a projective object $P_n$ and a surjection $\epsilon_n: P_n \rightarrow K_{n-1}$.
	Then we let $K_n \rightarrow P_n$ be the kernel of $\epsilon_n$, and define $d_n$ to be the composition $P_n \rightarrow K_n \rightarrow P_{n-1}$, i. e. $d_n = k_{n-1}\circ\epsilon_{n}$.
	Since $\epsilon_n$ is a surjection for every $n$, this gives us an exact complex
	\[\dots \overset{d}{\longrightarrow} P_2 \overset{d}{\longrightarrow} P_1 \overset{d}{\longrightarrow} P_0 \overset{\epsilon_0}{\longrightarrow} M \rightarrow 0.\]
	Since the $P_n$ are projective objects, it is a projective resolution of $M$.
\end{proof}

\begin{exercise}
	If $P$ is a complex of projectives with $P_i=0$ for $i<0$, then a map $\epsilon:P_0\rightarrow M$ giving a resolution for $M$ is the same as a chain map $\epsilon:P\rightarrow M$, where $M$ is considered as a complex concentrated in degree zero.
\end{exercise}

\begin{proof}
	Let $P$ be a complex such that together with a map $\epsilon: P_0 \rightarrow M$ it is a projective resolution of $M$.
	It suffices to show that the diagram
	\[
		\xymatrix{
			P_1 \ar[r] \ar[d]^d &0 \ar[d] \\
			P_0 \ar[r]^{\epsilon} &M
		}
	\]
	commutes, since every other square commutes trivially.
	But since $P_1 \overset{d}{\longrightarrow} P_0 \overset{\epsilon}{\longrightarrow} M \rightarrow $ has to be exact (as $P$ with $\epsilon$ is a projective resolution of $M$), we know that $\epsilon\circ d = 0$, so the diagram commutes.
	Therefore, we have a chain map $\epsilon: P \rightarrow M$.
\end{proof}

\begin{theorem}[Comparison Theorem]
	Let $P\overset{\epsilon}{\rightarrow} M$ be a projective resolution of $M$ and $f':M\rightarrow N$ a map in $\cat{A}$. Then for every resolution $Q\overset{\eta}{\rightarrow} N$ of $N$ there is a chain map $f:P\rightarrow Q$ lifting $f'$ in the sense that $\eta\circ f_0=f'\circ\epsilon$. The chain map $f$ is unique up to chain homotopy equivalence.
	\begin{align*}
		\xymatrix{
			\dots \ar[r] &P_2 \ar[r] \ar[d]^{\exists} &P_1 \ar[r] \ar[d]^{\exists} &P_0 \ar[r]^{\epsilon} \ar[d]^{\exists} &M \ar[r] \ar[d]^{f'} &0 \\
			\dots \ar[r] &Q_2 \ar[r] &Q_1 \ar[r] &Q_0 \ar[r]^{\eta} &N \ar[r] &0 
		}
	\end{align*}
\end{theorem}

\begin{proof}
	We will construct the $f_n$ and show their uniqueness by induction on $n$, thinking of $f_{-1}$ as $f'$.
	Inductively, suppose that $f_i$ has been constructed for $i\leq n$, such that $f_{i-1}d=df_i$.
	To construct $f_n$, we consider the $n$-cycles of $P$ and $Q$.
	If $n=-1$, we set $Z_{-1}(P) = M$ and $Z_{-1}(Q) = N$; if $n \geq 0$, $f_n$ induces a map $f'_n: Z_n(P) \rightarrow Z_n(Q)$ via $f_{n-1}d=df_n$.
	Therefore we have two diagrams with exact rows
	\[
		\vcenter{\xymatrix{
				\dots \ar[r]^d &P_{n+1} \ar[r]^d \ar[d]^{\exists} &Z_n(P) \ar[r] \ar[d]^{f'_n} &0 \\
				\dots \ar[r] &Q_{n+1} \ar[r]^d &Z_n(Q) \ar[r] &0
		}}
		\text{ and }
		\vcenter{\xymatrix{
				0 \ar[r] &Z_n(P) \ar[r] \ar[d]^{f'_n} &P_n \ar[r] \ar[d]^{f_n} &P_{n-1} \ar[d]^{f_{n-1}} \\
				0 \ar[r] &Z_n(Q) \ar[r] &Q_n \ar[r] &Q_{n-1}.
	}}
	\]
	Since $P_{n+1}$ is projective, we get a map $f_{n+1}: P_{n+1} \rightarrow Q_{n+1}$ such that $df_{n+1} = f'_nd = f_nd$.
	This proves the existence of the chain map $f: P \rightarrow Q$.
	
	We need to show that $f$ is unique up to chain homotopy.
	Let $g: P \rightarrow Q$ be another lift of $f'$ and let $h := f - g$.
	We will construct a chain contraction $\{s_n: P_n \rightarrow Q_{n+1}\}$ oh $h$ by induction on $n$.
	If $n<0$, then $P_0 = 0$, so we set $s_n = 0$.
	If $n=0$, then $h_0$ sends $P_0$ to $Z_0(Q)=d(Q_1)$, since $\eta h_0 = \epsilon(f' - f') = 0$.
	Since $P_0$ is projective, we can $h_0$ to the map $s_0: P_0 \rightarrow Q_1$ such that $h_0 = d s_0 = d s_0 + s_{-1} d$.
	Inductively, suppose that we are given maps $s_i$, $i<n$, such that $d s_{n-1} = h_{n-1} - s_{n-1} d$.
	Consider the map $h_n - s_{n-1} d: P_n \rightarrow Q_n$.
	Then
	\[d(h_n - s_{n-1} d) = d h_n - (h_{n-1} - s_{n-2} d) d = (dh - hd) + s_{n-2} d d = 0.\]
	Therefore, $h_n - s_{n-1}d$ lands in $Z_n(Q)$, a quotient of of quotient of $Q_{n+1}$.
	Since $P_n$ is projective, we get the desired map $s_n: P_n \rightarrow Q_{n+1}$ such that $d s_n = h_n - s_{n-1} d$.
\end{proof}

\begin{porism}
	In the proof we saw that the hypothesis that $P\rightarrow M$ is a projective resolution is too strong.
	It suffices to be given a chain complex
	$$\dots \rightarrow P_2 \rightarrow P_1 \rightarrow P_0 \rightarrow M \rightarrow 0$$
	with the $P_i$ projective.
	Then for every resolution $Q\rightarrow N$ of $N$, every map $M\rightarrow N$ lifts to a map $P\rightarrow Q$, which is unique up to chain homotopy.
	This stronger version of the Comparison Theorem will be used in section 2.7 to construct the external product for $\tor$.
\end{porism}

\begin{lemma}[Horseshoe Lemma]
	Suppose given a commutative diagram
	\begin{align*}
		\xymatrix{
			& & & &0 \ar[d] & \\
			\dots \ar[r] &P'_2 \ar[r] &P'_1 \ar[r] &P'_0 \ar[r]^{\epsilon'} &A' \ar[r] \ar[d]^{\iota_A} &0 \\
			& & & &A \ar[d]^{\pi_A} & \\
			\dots \ar[r] &P''_2 \ar[r] &P''_1 \ar[r] &P''_0 \ar[r]^{\epsilon''} &A'' \ar[r] \ar[d] &0 \\
			& & & &0 & 
		}
	\end{align*}
	where the column is exact and the rows are projective resolutions. Set $P_n:=P'_n\oplus P''_n$. Then the $P_n$ assemble to form a projective resolution $P$ of $A$ and the right hand column lifts to an exact sequence of complexes
	$$0\rightarrow P' \overset{\iota}{\rightarrow} P \overset{\pi}{\rightarrow} P'' \rightarrow 0,$$
	where $\iota_n:P'_n\rightarrow P_n$ and $\pi_n:P_n\rightarrow P''_n$ are the natural inclusion and projection, respectively.
\end{lemma}

\begin{proof}
	content...
\end{proof}

\begin{exercise}
	There are maps $\lambda_n: P''_n\rightarrow P*_{n-1}$ such that
	$$d=\begin{pmatrix}
			d' & \lambda \\
			0 & d'' \\
	\end{pmatrix},
	\quad\text{i. e.}\quad
	d'\begin{pmatrix}p'\\p''\end{pmatrix}=
		\begin{pmatrix}
			d'(p')+\lambda(p'') \\
			d''(p'')\\
		\end{pmatrix}$$
\end{exercise}