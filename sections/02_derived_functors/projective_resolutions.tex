\subsection{Projective Resolutions}

\begin{definition}
	An object $P$ in an abelian category is called \textit{projective}, if it satisfies the following universal lifting property:
	Given a surjection $g:B\rightarrow C$ and a map $\gamma:P\rightarrow C$, there is at least one map $\beta: P\rightarrow B$ such that $\gamma=g\circ\beta$.
	\begin{align*}
		\xymatrix{
			&P \ar[ld]_{\exists\beta} \ar[d]^\gamma \\
			B \ar[r]_g &C \ar[r] &0
		}
	\end{align*}
\end{definition}

\begin{remark}
	\begin{enumerate}[label=(\roman*)]
		\item Free $R$-modules are projective. \\
		Let $M$ be a free $R$-module with basis $\{x_i\vert i\in I\}$, $g:B\rightarrow C$ a surjection, $\gamma:M\rightarrow C$ a map.
		Since $g$ is surjective, we can find $b_i\in B$ such that $g(b_i)=\gamma(x_i)\in C$ for all $i \in I$.
		Then $M$ is projective with the map $\beta:M\rightarrow B$, $x_i\mapsto b_i$.\\
		\item If a direct sum of $R$-modules is projective, then so are its summands.\\
		Let $\bigoplus_{i\in I}M_i$ be projective, $g:B\rightarrow C$ a surjection, $\gamma_j:M_j \rightarrow C$ a map for some $j\in I$.
		Let $\iota_j:M_j\rightarrow \bigoplus_{i\in I}$ be the natural embedding. \\
		Consider the projection $\pi_j:\bigoplus_{i\in I}M_i\rightarrow M_j$. For $\gamma_j\circ\pi_j$, there is a lift $\beta_j:\bigoplus_{i\in I}M_i\rightarrow B$.
		\begin{align*}
			\xymatrix{
				\bigoplus_{i\in I}M_i \ar[r]^{\pi_j} \ar[d]^{\beta_j} \ar@{.>}[rd]^{\gamma_j\circ\pi_j} &M_j \ar[d]^{\gamma_j} \ar@/_2pc/[l]^{\iota_j} \\
				B \ar[r] &C \ar[r] &0
			}
		\end{align*}
		 Then $\beta_j\circ\iota_j: M_j\rightarrow B$ is a lift for $\gamma_j$, since for all $x \in M_j$:
		 $$\gamma_j(x) = (\gamma_j \circ (\pi_j \circ \iota_j))(x) = ((\gamma_j \circ \pi_j) \circ \iota_j)(x) = (\beta_j \circ \iota_j)(x).$$
	\end{enumerate}
\end{remark}

\begin{proposition}
	An $R$-module is projective if and only if it is a direct summand of a free $R$-module.
\end{proposition}

\begin{proof}
	Let $A$ be the a projective $R$-module. Let $R\{A\}$ denote the free $R$-module generated by the set of $A$. \\
	We have a surjection $R\{A\}\overset{\pi}{\longrightarrow}A$. Since $A$ is projective, we have a map $\iota:A\rightarrow R\{A\}$ such that $id_A=\pi\circ\iota$. \\
	We can restrict $\iota$ to the kernel of $\pi$ to get a short exact sequence
	$$0\rightarrow \ker A \rightarrow R\{A\} \rightarrow A \rightarrow 0.$$
	We can also restrict $\pi$ to the map $\pi':R\{\ker A\}\rightarrow \ker A$ and we have $\pi\circ\iota\vert_{\ker A}(x) = \pi'\circ\iota\vert_{\ker A}(x) = \id_{\ker A}(x)$ for all $x\in\ker A$. \\
	Therefore, the short exact sequence splits and we get an isomorphism $$R\{A\}\cong \ker A \oplus A.$$
	For the other implication, let $A$ be an direct summand of a free $R$-module $F$. Then $F$ is projective and so is $A$. 
\end{proof}

\begin{example}
	There are a lot of nice rings where every projective module is also free, e. g. $\Z$, fields and division rings.
	But this is not always the case.
	\begin{enumerate}[label=(\roman*)]
		\item Consider $R=R_1\times R_2$. Then $P=R_1\times 0$ and $0\times R_2$ are projective as they are direct summands of $R$. \\
		But $P$ is not free: \\
		Let $\varphi:R_1\times R_2 \overset{\cong}{\longrightarrow} R$. Consider $(0,1)\in R_1\times R_2$.
		Since $\varphi$ is an isomorphism, we know that $\varphi(0,1)\neq0$.
	\end{enumerate}
\end{example}

\begin{remark}
	The category $\cat{A}$ of finite abelian groups is an abelian category with no projective objects. \\
	We say that $\cat{A}$ \textit{has enough projectives} if for every object $A$ of $\cat{A}$ there is a surjection $P\rightarrow A$ with $P$ projective.
\end{remark}

\begin{lemma}
	$M$ is projective iff $\Hom_{\cat{A}}(M,-)$ is an exact functor, i. e. iff the sequence 
	$$0 \rightarrow \Hom(M,A) \rightarrow \Hom(M,B) \rightarrow \Hom(M,C) \rightarrow 0$$
	is exact for every exact sequence $0\rightarrow A \rightarrow B \rightarrow C \rightarrow 0$ in $\cat{A}$.
\end{lemma}

\begin{proof}
	
\end{proof}

\begin{exercise}
	A chain complex $P$ is a projective object in $\ch$ iff it is a split exact complex of projectives.
\end{exercise}

\begin{proof}
	content...
\end{proof}

\begin{exercise}
	If $\cat{A}$ has enough projectives, then so does the category $\ch(\cat{A})$ of chain complexes over $\cat{A}$.
\end{exercise}

\begin{proof}
	content...
\end{proof}

\begin{definition}
	Let $M$ be an object of $\cat{A}$. A \textit{left resolution} of $M$ is a complex $P$ with $P_i=0$ for $i<0$, together with a map $\epsilon:P_0\rightarrow M$ such that the augmented complex
	$$\dots \overset{d}{\longrightarrow} P_2 \overset{d}{\longrightarrow} P_1 \overset{d}{\longrightarrow} P_0 \overset{\epsilon}{\longrightarrow} M \longrightarrow 0$$
	is exact. It is a \textit{projective resolution} if each $P_i$ is projective.
\end{definition}

\begin{lemma}
	Every $R$-module $M$ has a projective resolution. More generally, if an abelian category $\cat{A}$ has enough projectives, then every object $M$ in $\cat{A}$ has a projective resolution.
\end{lemma}

\begin{exercise}
	If $P$ is a complex of projectives with $P_i=0$ for $i<0$, then a map $\epsilon:P_0\rightarrow M$ giving a resolution for $M$ is the same as a chain map $\epsilon:P\rightarrow M$, where $M$ is considered as a complex concentrated in degree zero.
\end{exercise}

\begin{theorem}[Comparison Theorem]
	Let $P\overset{\epsilon}{\rightarrow} M$ be a projective resolution of $M$ and $f':M\rightarrow N$ a map in $\cat{A}$. Then for every resolution $Q\overset{\eta}{\rightarrow} N$ of $N$ there is a chain map $f:P\rightarrow Q$ lifting $f'$ in the sense that $\eta\circ f_0=f'\circ\epsilon$. The chain map $f$ is unique up to chain homotopy equivalence.
	\begin{align*}
		\xymatrix{
			\dots \ar[r] &P_2 \ar[r] \ar[d]^{\exists} &P_1 \ar[r] \ar[d]^{\exists} &P_0 \ar[r]^{\epsilon} \ar[d]^{\exists} &M \ar[r] \ar[d]^{f'} &0 \\
			\dots \ar[r] &Q_2 \ar[r] &Q_1 \ar[r] &Q_0 \ar[r]^{\eta} &N \ar[r] &0 
		}
	\end{align*}
\end{theorem}

\begin{porism}
	In the proof we will see that the hypothesis that $P\rightarrow M$ is a projective resolution is too strong. It suffices to be given a chain complex
	$$\dots \rightarrow P_2 \rightarrow P_1 \rightarrow P_0 \rightarrow M \rightarrow 0$$
	with the $P_i$ projective. Then for every resolution $Q\rightarrow N$ of $N$, every map $M\rightarrow N$ lifts to a map $P\rightarrow Q$, which is unique up to chain homotopy.
	This stronger version of the Comparison Theorem will be used in section 2.7 to construct the external product for $\tor$.
\end{porism}

\begin{proof}
	content...
\end{proof}

\begin{lemma}[Horseshoe Lemma]
	Suppose given a commutative diagram
	\begin{align*}
		\xymatrix{
			& & & &0 \ar[d] & \\
			\dots \ar[r] &P'_2 \ar[r] &P'_1 \ar[r] &P'_0 \ar[r]^{\epsilon'} &A' \ar[r] \ar[d]^{\iota_A} &0 \\
			& & & &A \ar[d]^{\pi_A} & \\
			\dots \ar[r] &P''_2 \ar[r] &P''_1 \ar[r] &P''_0 \ar[r]^{\epsilon''} &A'' \ar[r] \ar[d] &0 \\
			& & & &0 & 
		}
	\end{align*}
	where the column is exact and the rows are projective resolutions. Set $P_n:=P'_n\oplus P''_n$. Then the $P_n$ assemble to form a projective resolution $P$ of $A$ and the right hand column lifts to an exact sequence of complexes
	$$0\rightarrow P' \overset{\iota}{\rightarrow} P \overset{\pi}{\rightarrow} P'' \rightarrow 0,$$
	where $\iota_n:P'_n\rightarrow P_n$ and $\pi_n:P_n\rightarrow P''_n$ are the natural inclusion and projection, respectively.
\end{lemma}

\begin{proof}
	content...
\end{proof}

\begin{exercise}
	There are maps $\lambda_n: P''_n\rightarrow P*_{n-1}$ such that
	$$d=\begin{pmatrix}
			d' & \lambda \\
			0 & d'' \\
	\end{pmatrix},
	\quad\text{i. e.}\quad
	d'\begin{pmatrix}p'\\p''\end{pmatrix}=
		\begin{pmatrix}
			d'(p')+\lambda(p'') \\
			d''(p'')\\
		\end{pmatrix}$$
\end{exercise}