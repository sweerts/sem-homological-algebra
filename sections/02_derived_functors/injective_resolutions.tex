\subsection{Injective Resolutions}

\begin{definition}
	An object $I$ in an abelian category $\cat{A}$ is \textit{injective} if it satisfies the following universal lifting property: \\
	Given an injection $f: A \rightarrow B$ and a map $\alpha:A\rightarrow I$, there exists at least one map $\beta:B\rightarrow I$ such that $\alpha=\beta\circ f$.
	\begin{align*}
		\xymatrix{
			0 \ar[r] &A \ar[r]^f \ar[d]^\alpha & B \ar[ld]^{\exists\beta} \\
			&I
		}
	\end{align*}
	We say that $\cat{A}$ \textit{has enough injectives} if for every object $A$ in $\cat{A}$ there is an injection $A\rightarrow I$ with $I$ injective. 
\end{definition}

\begin{note}
	If $\{I_\alpha\}$ is a family of injectives, then the product $\prod I_\alpha$ is also injective.
	The notion of injective module was invented by R. Baer in 1940, long before projective modules were thought of.
\end{note}

\begin{criterion}[Baer] \label{baer}
	A right $R$-module $E$ is injective if and only if for every right ideal $J$ of $R$, every map $J\rightarrow E$ can be extended to a map $R\rightarrow E$.
\end{criterion}

\begin{proof}
	If $E$ is injective and $J$ a right ideal of $R$, then every map $J \rightarrow E$ can be lifted to a map $R \rightarrow E$.
	
	Conversely, let $B$ be an $R$-module and let $A$ be a submodule of $B$.
	Let $\alpha: A \rightarrow E$ be a map.
	We need to extend $\alpha$ to a map $\alpha'': B \rightarrow E$.
	Let $\mathcal{E}$ be the set of all extensions of $\alpha$, partially ordered by $(\alpha_1, A_1) \leq (\alpha_2, A_2) \iff A_1 \subseteq A_2$ and $\alpha_2\vert_{A_1} = \alpha_1$.
	Zorn's lemma gives us a maximal extension $\alpha': A' \rightarrow E$.
	Suppose there is a $b \in B \setminus A'$.
	Then $J := \{r \in R \vert br \in A'\}$ is a right ideal of $R$.
	By assumption, we can extend the composition $J \overset{b}{\longrightarrow} A' \overset{\alpha'}{\longrightarrow} E$ to a map $f: R \rightarrow E$.
	Let $A'' = A' + bR$ be a submodule of $B$ and define $\alpha'': A'' \rightarrow E$ by
	\[\alpha''(a + br) := \alpha'(a) + f(r),\]
	where $a \in A'$ and $r \in R$.
	We need to show that this is a well-defined map.
	Assume that $a_0 + br_0 = a_1 + br_1$ with $a_0, a_1 \in A'$ and $r_0, r_1 \in R$.
	Then $b(r_0 - r_1) = a_1 - a_0 \in A'$, hence $r_0 - r_1 \in J$.
	Then we can see that
	\[\alpha'(a_1) - \alpha'(a_0) = \alpha'(a_1 - a_0) = \alpha'(b(r_0 - r_1)) = f(r_0 - r_1) = f(r_0) - f(r_1).\]
	Therefore, $\alpha'(a_1) + f(r_1) = \alpha'(a_0) + f(r_0)$, hence $\alpha''$ is well-defined.
	Let $a \in A'$.
	Then $\alpha''(a) = \alpha'(a)$, hence $\alpha''\vert_{A'} = \alpha_1$.
	Since we assumed that $A' \subsetneq B$, this contradicts the maximality of $\alpha'$.
	Therefore, $A' = B$ and we can conclude that $E$ is injective.
\end{proof}

\begin{exercise}
	Let $R=\Z/m$. Then $R$ is an injective $R$-module.
	Furthermore, $\Z/d$ is not an injective $R$-module when $d\vert m$ and some prime divides both $d$ and $m/d$
\end{exercise}

\begin{proof}
	We need to show that $\varphi: I \rightarrow \Z/m\Z$ can be extended to $\bar{\varphi}: \Z/m\Z \rightarrow \Z/m\Z$ for every ideal $I$ of $\Z/m\Z$.
	Let $I$ be an ideal of $\Z/m\Z$.
	Then $I$ is of the form $d\Z/m\Z \cong \Z/(\frac{m}{d})\Z$ for some $d \vert m$.
	Since \[\frac{m}{d}f(1) = f(\frac{m}{d}) = f(0) = 0,\]
	we know that $f(1) \in d\Z$, otherwise $\varphi$ would not be a well-defined morphism.
	Now we can extend $f$ to a morphism $\bar{f}: \Z/m\Z \rightarrow \Z/m\Z$ via $1 \mapsto \frac{f(1)}{d}$.
	This is well-defined, since $d \vert f(1)$.
	By Baer's Criterion, $\Z/m\Z$ is an injective $\Z/m\Z$-module.
	
	Let $d\vert m$ such that there is a prime with $p \vert d$ and $p \vert \frac{m}{d}$.
	Let $f: \Z/p \rightarrow \Z/m$ be the map $1 \mapsto \frac{m}{p}$.
	Let $g: \Z/p \rightarrow \Z/d$ be the map $1 \mapsto \frac{d}{p}$.
	Since $\ker(f) = \{x \in \Z/p : p \vert x\} = \{0\}$ and $\ker(g) = \{x \in \Z/p : p \vert x\} = \{0\}$, we know that $f$ and $g$ are injective.
	Now consider any map $h: \Z/m \rightarrow \Z/d$.
	Then any multiple of $d$ in $\Z/m$ gets send to $0$, since $h(d\cdot a) = d \cdot h(a) = 0$ in $\Z/d$.
	We also know that $p \vert \frac{m}{d}$.
	Then $d \vert \frac{m}{p}$, and therefore $h\circ f= 0$, which proves that $h\circ f \neq g$ for all $h$.
	Therefore $\Z/d$ cannot be an injective.
\end{proof}

\begin{corollary}
	Suppose that $R=\Z$, or more generally that $R$ is a principle ideal domain. An $R$-module $A$ is injective iff it is divisible, i. e., for every $r\neq0$ in $R$ and every $a\in A, a=br$ for some $b\in A$.
\end{corollary}

\begin{proof}
	Let $R$ be a principle ideal domain and let $A$ be an $R$-module.
	
	Suppose that $A$ is injective.
	Let $r \in R\setminus\{0\}$ an let $a \in A$.
	Consider the homomorphism $f: R \rightarrow R$, $1 \mapsto r$.
	Since $r\neq0$ and $R$ is a principal ideal domain, we know that $f(x) = rx  = 0 \iff x = 0$.
	Hence, $f$ is injective.
	Now consider the map $\alpha: R \rightarrow A$, $1 \mapsto a$.
	We can lift $\alpha$ to the map $\beta: R \rightarrow A$, i. e. $\alpha = \beta\circ f$.
	We then know
	\[a = \alpha(1) = \beta(f(1)) = \beta(r) = r \cdot \beta(1).\]
	
	Conversely, let $J$ be any ideal of $R$ and let $f: J \rightarrow A$ be any map.
	Since $R$ is a principle ideal domain, there is a generator $r \in R$ of $J$.
	If $r=0$, the zero map $R \rightarrow A$ extends the map $0 = J \rightarrow A$.
	If $r \neq 0$, then there is a $b \in A$ such that $f(b) = br$, since $A$ is divisible.
	Consider $\beta: R \rightarrow A$, $1 \mapsto b$.
	Then \[\beta(rk) = b(rk) = (br)k = f(r)k = f(rk),\] hence $\beta$ extends $f$.
	By \hyperref[baer]{Baer's Criterion}, $A$ is injective.
\end{proof}

\begin{example}
	The divisible abelian groups $\Q$ and $\Z_{p^\infty}=\Z[\frac{1}{p}]/\Z$ are injective ($\Z[\frac{1}{p}]$ is the group of rational numbers of the form $a/p^n,n\geq 1$). Every injective abelian group is a direct sum of these. In particular, the injective abelian group $\Q/\Z$ is isomorphic to $\bigoplus\Z_{p^\infty}$.
\end{example}

The category $\ab$ has enough injectives.
Consider an abelian group $A$. Let $I(A)$ be the product of copies of the injective group $\Q/\Z$, indexed by $\Hom_{\ab}(A,\Q/\Z)$.
Then $I(A)$ is injective, being a product of injectives, and there is a canonical map $e_A:A\rightarrow I(A)$.
This is the desired injection of $A$ into an injective.

\begin{exercise}
	$e_A$ is an injection.
\end{exercise}

\begin{proof}
	content...
\end{proof}

\begin{exercise}
	An abelian group $A$ is zero iff $\Hom_{\ab}(A,\Q/\Z)=0$.
\end{exercise}

\begin{proof}
	content...
\end{proof}

\begin{remark}
	It is easily verified, that if $\cat{A}$ is an abelian category, then $\cat{A}^{op}$ is also abelian.
	The definition of injective is dual to the definition of projective, so the following result are easily deduced by arguing in $\cat{A}^{op}$.
\end{remark}

\begin{proof}
	Let $\cat{A}$ be an abelian category.
	For every diagram
	\[
		\xymatrix{
			A \ar[r]^f &B \ar@<-.5ex>[r]_g \ar@<.5ex>[r]^{g'} &C \ar[r]^h &D
		}
	\]
	in $\cat{A}^{op}$ we have $f \in \Hom_{\cat{A}}(B, A)$, $g,g' \in \Hom_{\cat{A}}(C, B)$ and $h \in \Hom_{\cat{A}}(D,C)$.
	Since $\cat{A}$ is an $\ab$-category, we know that $f(g+g')h = fgh + fg'h$ in $\cat{A}$.
	Therefore, we know $h(g+g')h = hgf + hg'f$ in $\cat{A}^{op}$, and $\cat{A}^{op}$ is an $\ab$-category.
	
	We know that $\Hom_{\cat{A}^{op}}(0,X) = \Hom_{\cat{A}}(X,0)$ for all $X \in \Ob(\cat{A}^{op})$ and $\Hom_{\cat{A}^{op}}(X,0) = \Hom_{\cat{A}}(0,X)$ for all $X \in \Ob(\cat{A}^{op})$.
	Therefore, $0$ is a zero object in $\cat{A}^{op}$.
	Let $B,C$ be two objects of $\cat{A}^{op}$.
	Since $\cat{A}$ is an additive category, the coproduct $B\coprod C$ exists in $\cat{A}$ and coincides with the product $B\times C$.
	By duality, the product and coproduct of $B$ and $C$ exist in $\cat{A}^{op}$ and they coincide.
	Therefore, $\cat{A}^{op}$ is an additive category.
	
	Let $f \in \Hom_{\cat{A}^{op}}(B,C)$ be a morphism.
	Then $f$ is a morphism in $\Hom_{\cat{A}}(C,B)$ and has a kernel and cokernel in $\cat{A}$.
	By duality, $f$ has a cokernel and kernel in $\cat{A}^{op}$.
	Let $f$ be a monomorphism in $\cat{A}^{op}$.
	Then $f$ is an epimorphism in $\cat{A}$ is the cokernel of its kernel.
	By duality, $f$ is the kernel of its cokernel in $\cat{A}^{op}$.
	Let $f$ be an epimorphism in $\cat{A}^{op}$.
	Then $f$ is a monomorphism in $\cat{A}$ is the kernel of its cokernel.
	By duality, $f$ is the cokernel of its kernel in $\cat{A}^{op}$.
	Therefore, $\cat{A}^{op}$ is an abelian category.
\end{proof}

\begin{lemma}
	Let $I$ be an object in an abelian category $\cat{A}$. The following statements are equivalent:
	\begin{enumerate}[label=(\roman*)]
		\item $I$ is injective in $\cat{A}$.
		
		\item $I$ is projective in $\cat{A}^{op}$.
		
		\item The contravariant functor $\Hom_\cat{A}(-,I)$ is exact, i. e. it takes short exact sequences in $\cat{A}$ to short exact sequences in $\ab$.
	\end{enumerate}
\end{lemma}

\begin{proof}
	Let $I$ be an injective object in $\cat{A}$.
	It is easy to see, that $I$ must be a projective object in $\cat{A}^{op}$.
	\[
		\vcenter{\xymatrix{
			0 \ar[r] &A \ar[r]^f \ar[d]^\alpha & B \ar[ld]^{\exists\beta} \\
			&I
		}}
		\quad\text{in $\cat{A}$ becomes}\quad
		\vcenter{\xymatrix{
			0  &A \ar[l]   & B \ar[l]^f  \\
			&I \ar[u]^\alpha \ar[ru]_{\exists\beta}
		}}
	\quad\text{in $\cat{A}^{op}$}
	\]
	Similarly, a projective object in $\cat{A}^{op}$ is an injective object in $(\cat{A}^{op})^{op} = \cat{A}^{op}$.
	
	The third statement can also be viewed in $\cat{A}^{op}$.
	It is equivalent to the statement
\end{proof}

\begin{definition}
	Let $M$ be an object of $\cat{A}$.
	A \textit{right resolution} of $M$ is a cochain complex $I$ with $I^i=0$ for $i<0$ and a map $M\rightarrow I^0$ such that the augmented complex
	$$0 \rightarrow M \rightarrow I^0 \overset{d}{\rightarrow} I^1 \overset{d}{\rightarrow} I^2 \overset{d}{\rightarrow} \dots$$
	is exact.
	This is the same as a cochain map $M\rightarrow I$, where $M$ is considered as a complex concentrated in degree $0$.
	It is called an \textit{injective resolution} if each $I^i$ is injective.
\end{definition}

\begin{lemma}
	If the abelian category $\cat{A}$ has enough injectives, then every object in $\cat{A}$ has an injective resolution.
\end{lemma}

\begin{proof}
	content...
\end{proof}

\begin{theorem}[Comparison Theorem]
	Let $N\rightarrow I$ be an injective resolution of $N$ and $f':M\rightarrow N$ a map in $\cat{A}$. Then for every resolution $M\rightarrow E$ there is a cochain map $F:E\rightarrow I$ lifting $f'$.
	The map $f$ is unique up to cochain homotopy equivalence.
	\begin{align*}
		\xymatrix{
			0 \ar[r] &M \ar[r] \ar[d]^{f'} &E^0 \ar[r] \ar[d]^{\exists} &E^1 \ar[r] \ar[d]^{\exists} &E^2 \ar[r] \ar[d]^{\exists} &\dots \\
			0 \ar[r] &N \ar[r] &I^0 \ar[r] &I^1 \ar[r]^{\eta} &I^2 \ar[r] &\dots
		}
	\end{align*}
\end{theorem}

\begin{exercise}
	$I$ is an injective object in the category of chain complexes iff $I$ is a split exact complex of injectives. \\
	If $\cat{A}$ has enough injectives, then $\ch(\cat{A})$ of chain complexes over $\cat{A}$.
\end{exercise}

\begin{lemma}
	For every right $R$-module $M$, the natural map
	$$\tau:\Hom_{\ab}(M,A)\rightarrow\Hom_{mod-R}(M,\Hom_{\ab}(R,A))$$
	is an isomorphism, where $(\tau f)(m)$ is the map $r\mapsto f(mr)$.
\end{lemma}

\begin{proof}
	content...
\end{proof}

\begin{definition}
	A pair of functors $L:\cat{A}\rightarrow\cat{B}$ and $R:\cat{B}\rightarrow\cat{A}$ are \textit{adjoint} if there is a natural bijection for all $A$ in $\cat{A}$ and $B$ in $\cat{B}$:
	$$\tau = \tau_{AB}:\Hom_{\cat{B}}(L(A),B) \overset{\cong}{\rightarrow} \Hom_{\cat{A}}(A,R(B)).$$
	Here, "natural" means that for all $f:A\rightarrow A'$ in $\cat{A}$ and $g:B\rightarrow B'$ in $\cat{B}$ the following diagram commutes:
	\begin{align*}
		\xymatrix{
			\Hom_{\cat{B}}(L(A'),B) \ar[r]^{Lf^*} \ar[d]^{\tau} &\Hom_{\cat{B}}(L(A),B) \ar[r]^{g_*} \ar[d]^{\tau} &\Hom_{\cat{B}}(L(A),B') \ar[d]^{\tau} \\
			\Hom_{\cat{A}}(A',R(B)) \ar[r]^{f^*} &\Hom_{\cat{A}}(A,R(B)) \ar[r]^{Rg_*} &\Hom_{\cat{A}}(A,R(B')).
		}
	\end{align*}
	We call $L$ the \textit{left adjoint} and $R$ the \textit{right adjoint} of this pair.
	The above lemma states that the forgetful functor from $mod-R$ to $\ab$ has $\Hom_{\ab}(R,-)$ as its right adjoint.
\end{definition}

\begin{proposition}\label{PresInj}
	If an additive functor $R:\cat{B}\rightarrow\cat{A}$ is right adjoint to an exact functor $L:\cat{A}\rightarrow\cat{B}$ and $I$ is an injective object of $\cat{B}$, then $R(I)$ is an injective object of $\cat{A}$. (We say that $R$ preserves injectives.)
	
	Dually, if an additive functor $L:\cat{A}\rightarrow\cat{B}$ is left adjoint to an exact functor $R:\cat{B}\rightarrow\cat{A}$ and $P$ is a projective object of $\cat{A}$, then $L(P)$ is a projective object of $\cat{B}$. (We say that $L$ preserves projectives.)
\end{proposition}

\begin{proof}
	content...
\end{proof}

\begin{corollary}
	If $I$ is an injective abelian group, then $\Hom_{\ab}(R,I)$ is an injective $R$-module.
\end{corollary}

\begin{exercise}
	If $M$ is an $R$-module, let $I(M)$ be the product of copies of $I_0=\Hom_{\ab}(R,\Q/\Z)$, indexed by the set $\Hom_R(M,I_0)$.
	There is a canonical map $e_M:M\rightarrow I(M)$, which is an injection.
	Being a product of injectives, $I(M)$ is an injective, so this proves that $R-mod$ has enough injectives.
	An important consequence of this is that every $R$-module has an injective resolution.
\end{exercise}

\begin{proof}
	content...
\end{proof}

\begin{example}
	The category $\sheaves(X)$ of abelian group sheaves on a topological space $X$ has enough injectives.
	To see this, we need two constructions.
	The \textit{stalk} of a sheaf $\mathcal{F}$ at a point $x\in X$ is the abelian group $\mathcal{F}_x = \underset{\longrightarrow}{\lim}\{\mathcal{F}(U):x\in U\}$.
	"Stalk at $x$" is an exact functor from $\sheaves(X)$ to $\ab$.
	If $A$ is an abelian group, the \textit{skyscraper sheaf} $x_*A$ at the point $x\in X$ is defined to be the presheaf
	$$(x_*A)(U)=\left\{\begin{array}{ll}A & \text{if }x\in U \\ 0 & \text{otherwise}\end{array}\right.$$
\end{example}

\begin{exercise}
	$x_*A$ is a sheaf and 
	$$\Hom_{\ab}(\mathcal{F}_x,A) \cong \Hom_{\sheaves(X)}(\mathcal{F},x_*A)$$
	for every sheaf $\mathcal{F}$.
	Furthermore, if $A_x$ is an injective abelian group, then $x_*(A_x)$ is an injective object in $\sheaves(X)$ for each $x$, and $\prod_{x\in X} x_*(A_*)$ is also injective.
\end{exercise}

\begin{proof}
	content...
\end{proof}

\begin{example}
	Let $I$ be a small category and $\cat{A}$ an abelian category.
	If the product of any set of objects exists in $\cat{A}$ ($\cat{A}$ is complete) and $\cat{A}$ has enough injectives, we will show that the functor category $\cat{A}^I$ has enough injectives.
	For each $k$ in $I$, the $k^\text{th}$ coordinate $A\mapsto A(k)$ is an exact functor from $\cat{A}^I$ to $A$. Given $A$ in $\cat{A}$, define the functor $k_*A:I\rightarrow\cat{A}$ by sending $i\in I$ to
	$$k_*A(i)=\prod_{\Hom_{I}(i,k)}A.$$
	If $\eta:i\rightarrow j$ is a map in $I$, the map $k_*A(i)\rightarrow k_*A(j)$ is determined by the index map $\eta^*:\Hom(j,k)\rightarrow \Hom(i,k)$.
	That is, the coordinate $k_*A(i)\rightarrow A$ of this map corresponding to $\varphi\in\Hom(j,k)$ is the projection of $k_*A(i)$ onto the factor corresponding to $\eta^*\varphi=\varphi\eta\in\Hom(i,k)$.
	If $f:A\rightarrow B$ is a map $k_*A\rightarrow k_*B$ defined slotwise.
	In this way, $k_*$ becomes an additive functor from $\cat{A}$ to $\cat{A'}$, assuming that $\cat{A}$ has enough products for $k_*A$ to be defined.
\end{example}

\begin{exercise}
	Let $\cat{A}$ be complete and with enough injectives.
	Then $k_*$ is right adjoint to the $k^\text{th}$ coordinate functor, so that $k_*$ preserves injectives by \ref{PresInj}.
	Furthermore, $\mathcal{A}^I$ has enough injectives.
\end{exercise}

\begin{exercise}
	Let $\mathcal{A}$ be cocomplete and have enough projectives.
	Then $\mathcal{A}^I$ has enough projectives.
\end{exercise}