\subsection{Injective Resolutions}

\begin{definition}
	An object $I$ in an abelian category $\cat{A}$ is \textit{injective} if it satisfies the following universal lifting property: \\
	Given an injection $F:A\rightarrow B$ and a map $\alpha:A\rightarrow I$, there exists at least one map $\beta:B\rightarrow I$ such that $\alpha=\beta\circ f$.
	\begin{align*}
		\xymatrix{
			0 \ar[r] &A \ar[r]^f \ar[d]^\alpha & B \ar[ld]^{\exists\beta} \\
			&I
		}
	\end{align*}
	We say that $\cat{A}$ \textit{has enough injectives} if for every object $A$ in $\cat{A}$ there is an injection $A\rightarrow I$ with $I$ injective. Note that if $\{I_\alpha\}$ is a family of injectives, then the product $\prod I_\alpha$ is also injective.
	The notion of injective module was invented by R. Baer in 1940, long before projective modules were thought of.
\end{definition}

\begin{criterion}[Baer]
	A right $R$-module $E$ is injective iff for every right ideal $J$ of $R$, every map $J\rightarrow E$ can be extended to a map $R\rightarrow E$.
\end{criterion}

\begin{proof}
	content...
\end{proof}

\begin{exercise}
	Let $R=\Z/m$. Then $R$ is an injective $R$-module.
	Furthermore, $\Z/d$ is not an injective $R$-module when $d\vert m$ and some prime divides both $d$ and $m/d$
\end{exercise}

\begin{proof}
	content...
\end{proof}

\begin{corollary}
	Suppose that $R=\Z$, or more generally that $R$ is a principle ideal domain. An $R$-module $A$ is injective iff it is divisible, i. e., for every $r\neq0$ in $R$ and every $a\in A, a=br$ for some $b\in A$.
\end{corollary}

\begin{example}
	The divisible abelian groups $\Q$ and $\Z_{p^\infty}=\Z[\frac{1}{p}]/\Z$ are injective ($\Z[\frac{1}{p}]$ is the group of rational numbers of the form $a/p^n,n\geq 1$). Every injective abelian group is a direct sum of these. In particular, the injective abelian group $\Q/\Z$ is isomorphic to $\bigoplus\Z_{p^\infty}$.
\end{example}

The category $\ab$ has enough injectives.
Consider an abelian group $A$. Let $I(A)$ be the product of copies of the injective group $\Q/\Z$, indexed by $\Hom_{\ab}(A,\Q/\Z)$.
Then $I(A)$ is injective, being a product of injectives, and there is a canonical map $e_A:A\rightarrow I(A)$.
This is the desired injection of $A$ into an injective.

\begin{exercise}
	$e_A$ is an injection.
\end{exercise}

\begin{proof}
	content...
\end{proof}

\begin{exercise}
	An abelian group $A$ is zero iff $\Hom_{\ab}(A,\Q/\Z)=0$.
\end{exercise}

\begin{proof}
	content...
\end{proof}

\begin{remark}
	It is easily verified, that if $\cat{A}$ is an abelian category, then $\cat{A}^{op}$ is also abelian.
	The definition of injective is dual to the definition of projective, so the following result are easily deduced by arguing in $\cat{A}^{op}$.
\end{remark}

\begin{lemma}
	Let $I$ be an object in an abelian category $\cat{A}$.
	\begin{enumerate}[label=(\roman*)]
		\item $I$ is injective in $\cat{A}$.
		
		\item $I$ is projective in $\cat{A}^{op}$.
		
		\item The contravarian functor $\Hom_\cat{A}(-,I)$ is exact, i. e. it takes short exact sequences in $\cat{A}$ to short exact sequences in $\ab$.
	\end{enumerate}
\end{lemma}

\begin{proof}
	content...
\end{proof}

\begin{definition}
	Let $M$ be an object of $\cat{A}$.
	A \textit{right resolution} of $M$ is a cochain complex $I$ with $I^i=0$ for $i<0$ and a map $M\rightarrow I^0$ such that the augmented complex
	$$0 \rightarrow M \rightarrow I^0 \overset{d}{\rightarrow} I^1 \overset{d}{\rightarrow} I^2 \overset{d}{\rightarrow} \dots$$
	is exact.
	This is the same as a cochain map $M\rightarrow I$, where $M$ is considered as a complex concentrated in degree $0$.
	It is called an \textit{injective resolution} if each $I^i$ is injective.
\end{definition}

\begin{lemma}
	If the abelian category $\cat{A}$ has enough injectives, then every object in $\cat{A}$ has an injective resolution.
\end{lemma}

\begin{proof}
	content...
\end{proof}

\begin{theorem}[Comparison Theorem]
	Let $N\rightarrow I$ be an injective resolution of $N$ and $f':M\rightarrow N$ a map in $\cat{A}$. Then for every resolution $M\rightarrow E$ there is a cochain map $F:E\rightarrow I$ lifting $f'$.
	The map $f$ is unique up to cochain homotopy equivalence.
	\begin{align*}
		\xymatrix{
			0 \ar[r] &M \ar[r] \ar[d]^{f'} &E^0 \ar[r] \ar[d]^{\exists} &E^1 \ar[r] \ar[d]^{\exists} &E^2 \ar[r] \ar[d]^{\exists} &\dots \\
			0 \ar[r] &N \ar[r] &I^0 \ar[r] &I^1 \ar[r]^{\eta} &I^2 \ar[r] &\dots
		}
	\end{align*}
\end{theorem}

\begin{exercise}
	$I$ is an injective object in the category of chain complexes iff $I$ is a split exact complex of injectives. \\
	If $\cat{A}$ has enough injectives, then $\ch(\cat{A})$ of chain complexes over $\cat{A}$.
\end{exercise}

\begin{lemma}
	For every right $R$-module $M$, the natural map
	$$\tau:\Hom_{\ab}(M,A)\rightarrow\Hom_{mod-R}(M,\Hom_{\ab}(R,A))$$
	is an isomorphism, where $(\tau f)(m)$ is the map $r\mapsto f(mr)$.
\end{lemma}

\begin{proof}
	content...
\end{proof}

\begin{definition}
	A pair of functors $L:\cat{A}\rightarrow\cat{B}$ and $R:\cat{B}\rightarrow\cat{A}$ are \textit{adjoint} if there is a natural bijection for all $A$ in $\cat{A}$ and $B$ in $\cat{B}$:
	$$\tau = \tau_{AB}:\Hom_{\cat{B}}(L(A),B) \overset{\cong}{\rightarrow} \Hom_{\cat{A}}(A,R(B)).$$
	Here, "natural" means that for all $f:A\rightarrow A'$ in $\cat{A}$ and $g:B\rightarrow B'$ in $\cat{B}$ the following diagram commutes:
	\begin{align*}
		\xymatrix{
			\Hom_{\cat{B}}(L(A'),B) \ar[r]^{Lf^*} \ar[d]^{\tau} &\Hom_{\cat{B}}(L(A),B) \ar[r]^{g_*} \ar[d]^{\tau} &\Hom_{\cat{B}}(L(A),B') \ar[d]^{\tau} \\
			\Hom_{\cat{A}}(A',R(B)) \ar[r]^{f^*} &\Hom_{\cat{A}}(A,R(B)) \ar[r]^{Rg_*} &\Hom_{\cat{A}}(A,R(B')).
		}
	\end{align*}
	We call $L$ the \textit{left adjoint} and $R$ the \textit{right adjoint} of this pair.
	The above lemma states that the forgetful functor from $mod-R$ to $\ab$ has $\Hom_{\ab}(R,-)$ as its right adjoint.
\end{definition}

\begin{proposition}\label{PresInj}
	If an additive functor $R:\cat{B}\rightarrow\cat{A}$ is right adjoint to an exact functor $L:\cat{A}\rightarrow\cat{B}$ and $I$ is an injective object of $\cat{B}$, then $R(I)$ is an injective object of $\cat{A}$. (We say that $R$ preserves injectives.)
	
	Dually, if an additive functor $L:\cat{A}\rightarrow\cat{B}$ is left adjoint to an exact functor $R:\cat{B}\rightarrow\cat{A}$ and $P$ is a projective object of $\cat{A}$, then $L(P)$ is a projective object of $\cat{B}$. (We say that $L$ preserves projectives.)
\end{proposition}

\begin{proof}
	content...
\end{proof}

\begin{corollary}
	If $I$ is an injective abelian group, then $\Hom_{\ab}(R,I)$ is an injective $R$-module.
\end{corollary}

\begin{exercise}
	If $M$ is an $R$-module, let $I(M)$ be the product of copies of $I_0=\Hom_{\ab}(R,\Q/\Z)$, indexed by the set $\Hom_R(M,I_0)$.
	There is a canonical map $e_M:M\rightarrow I(M)$, which is an injection.
	Being a product of injectives, $I(M)$ is an injective, so this proves that $R-mod$ has enough injectives.
	An important consequence of this is that every $R$-module has an injective resolution.
\end{exercise}

\begin{proof}
	content...
\end{proof}

\begin{example}
	The category $\sheaves(X)$ of abelian group sheaves on a topological space $X$ has enough injectives.
	To see this, we need two constructions.
	The \textit{stalk} of a sheaf $\mathcal{F}$ at a point $x\in X$ is the abelian group $\mathcal{F}_x = \underset{\longrightarrow}{\lim}\{\mathcal{F}(U):x\in U\}$.
	"Stalk at $x$" is an exact functor from $\sheaves(X)$ to $\ab$.
	If $A$ is an abelian group, the \textit{skyscraper sheaf} $x_*A$ at the point $x\in X$ is defined to be the presheaf
	$$(x_*A)(U)=\left\{\begin{array}{ll}A & \text{if }x\in U \\ 0 & \text{otherwise}\end{array}\right.$$
\end{example}

\begin{exercise}
	$x_*A$ is a sheaf and 
	$$\Hom_{\ab}(\mathcal{F}_x,A) \cong \Hom_{\sheaves(X)}(\mathcal{F},x_*A)$$
	for every sheaf $\mathcal{F}$.
	Furthermore, if $A_x$ is an injective abelian group, then $x_*(A_x)$ is an injective object in $\sheaves(X)$ for each $x$, and $\prod_{x\in X} x_*(A_*)$ is also injective.
\end{exercise}

\begin{proof}
	content...
\end{proof}

\begin{example}
	Let $I$ be a small category and $\cat{A}$ an abelian category.
	If the product of any set of objects exists in $\cat{A}$ ($\cat{A}$ is complete) and $\cat{A}$ has enough injectives, we will show that the functor category $\cat{A}^I$ has enough injectives.
	For each $k$ in $I$, the $k^\text{th}$ coordinate $A\mapsto A(k)$ is an exact functor from $\cat{A}^I$ to $A$. Given $A$ in $\cat{A}$, define the functor $k_*A:I\rightarrow\cat{A}$ by sending $i\in I$ to
	$$k_*A(i)=\prod_{\Hom_{I}(i,k)}A.$$
	If $\eta:i\rightarrow j$ is a map in $I$, the map $k_*A(i)\rightarrow k_*A(j)$ is determined by the index map $\eta^*:\Hom(j,k)\rightarrow \Hom(i,k)$.
	That is, the coordinate $k_*A(i)\rightarrow A$ of this map corresponding to $\varphi\in\Hom(j,k)$ is the projection of $k_*A(i)$ onto the factor corresponding to $\eta^*\varphi=\varphi\eta\in\Hom(i,k)$.
	If $f:A\rightarrow B$ is a map $k_*A\rightarrow k_*B$ defined slotwise.
	In this way, $k_*$ becomes an additive functor from $\cat{A}$ to $\cat{A'}$, assuming that $\cat{A}$ has enough products for $k_*A$ to be defined.
\end{example}

\begin{exercise}
	Let $\cat{A}$ be complete and with enough injectives.
	Then $k_*$ is right adjoint to the $k^\text{th}$ coordinate functor, so that $k_*$ preserves injectives by \ref{PresInj}.
	Furthermore, $\mathcal{A}^I$ has enough injectives.
\end{exercise}

\begin{exercise}
	Let $\mathcal{A}$ be cocomplete and have enough projectives.
	Then $\mathcal{A}^I$ has enough projectives.
\end{exercise}