\section{Introduction}
This talk is based on \cite{Wei94}.

\begin{note}[Snake Lemma]\label{snake_lemma}
	Let
	\begin{align*}
		\begin{CD}
			0 @>>> A @>f>> B @>g>> C @>>> 0 \\
			@.   @VaVV   @VbVV    @VcVV    @. \\
			0 @>>> A' @>f'>> B' @>g'>> C' @>>> 0
		\end{CD}
	\end{align*}
	be a commutative diagram with exact rows. \\
	Then there is an exact sequence
	$$0 \rightarrow \ker(a) \rightarrow \ker(b) \rightarrow \ker(c) \rightarrow \coker(a) \rightarrow \coker(b) \rightarrow \coker(c) \rightarrow 0.$$
\end{note}

\begin{definition}[Mapping Cone]
	Let $f:B \rightarrow C$ be a map of chain complexes.
	The \textit{mapping cone} of $f$ is the chain complex whose degree $n$ part is $B_{n-1} \oplus C_n$, i. e.
	\[\dots \rightarrow B_{n} \oplus C_{n+1} \rightarrow B_{n-1} \oplus C_n \rightarrow B_{n-2} \oplus C_{n-1} \rightarrow \dots\]
	The differential is given by
	\[d(b,c) = (-d(b), d(c) - f(c)), \quad (b\in B_{n-1}, c\in C_n).\]
	There is a short exact sequence $0 \rightarrow C \rightarrow \cone(f) \rightarrow B[-1] \rightarrow 0.$ \\
	The mapping cone of $id_C$ (denoted by $\cone(C)$) is \textit{split exact}, i. e.
	\begin{enumerate}[label=(\roman*)]
		\item there are maps $s_n:C_n\rightarrow C_{n+1}$ such that $d=dsd$ ($C$ is \textit{split}),
		\item $C$ is exact as a sequence (\textit{acyclic}).
	\end{enumerate} 
\end{definition}