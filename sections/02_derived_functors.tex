\section{Derived Functors}

\subsection{$\delta$-Functors}

\begin{definition}
	A (covariant) homological $\delta$-functor between $\mathcal{A}$ and $\mathcal{B}$ is a collection of additive functors $T_n:\mathcal{A}\rightarrow\mathcal{B}$ for $n\geq0$, together with morphisms
	$$\delta_n:T_n(C)\rightarrow T_{n-1}(A)$$
	defined for each short exact sequence $0\rightarrow A \rightarrow B \rightarrow C \rightarrow 0$ in $\mathcal{A}$.
	We will assume that $T_n = 0$ for $n\leq 0$. \\
	The following two conditions are imposed:
	\begin{enumerate}[label=\arabic*.]
		\item For each short exact sequence $0 \rightarrow A \rightarrow B \rightarrow C \rightarrow 0$, there is a long exact sequence
		$$\dots T_{n+1}(C) \overset{\delta}{\rightarrow} T_n(A) \rightarrow T_n(B) \rightarrow T_n(C) \overset{\delta}{\rightarrow} T_{n-1}(A) \dots$$
		In particular, $T_0$ is right exact, and $T^0$ is left exact.
		
		\item For each morphism of short exact sequences from $0 \rightarrow A' \rightarrow B' \rightarrow C' \rightarrow 0$ to $0 \rightarrow A \rightarrow B \rightarrow C \rightarrow 0$, the $\delta$'s give a commutative diagram
		\begin{align*}
			\begin{gathered}
				\xymatrixcolsep{5pc}\xymatrixcolsep{2pc}\xymatrix{
				T_n(C') \ar[r]^{\delta} \ar[d] & T_{n-1}(A') \ar[d] \\
				T_n(C) \ar[r]^{\delta} & T_{n-1}(A). }
			\end{gathered}
		\end{align*}
	\end{enumerate}
\end{definition}

\begin{example}
	Homology gives a homological $\delta$-functor $H_*$ from $\cat{Ch_{\geq0}}{\mathcal{A}}$ to $\mathcal{A}$.
\end{example}

\begin{exercise}
	Let $\mathcal{S}$ be the category of short exact sequences
	\begin{equation}
		0 \rightarrow A \rightarrow B \rightarrow C \rightarrow 0 \tag{\textasteriskcentered}
	\end{equation}
	in $\mathcal{A}$. \\
	Then $\delta_i$ is a natural transformation from the functor sending (\textasteriskcentered) to $T_i(C)$ to the functor sending (\textasteriskcentered) to $T_{i-1}(A)$.
\end{exercise}

\begin{proof}
	Let 
\end{proof}

\subsection{Projective Resolutions}

\subsection{Injective Resolutions}

\subsection{Left Derived Functors}

\subsection{Right Derived Functors}